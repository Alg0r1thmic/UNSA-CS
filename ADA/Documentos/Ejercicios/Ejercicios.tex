\documentclass[a4paper,12pt]{article}
\usepackage[utf8]{inputenc}
\usepackage[spanish]{babel}
\usepackage{color}
\usepackage{parskip}
\usepackage{graphicx}
\usepackage{multirow}
\usepackage{listings}
\usepackage{vmargin}
\graphicspath{ {imagenes/} }
\definecolor{mygreen}{rgb}{0,0.6,0}
\definecolor{lbcolor}{rgb}{0.9,0.9,0.9}
\usepackage{epstopdf}


\setpapersize{A4}
\setmargins{2.5cm}       % margen izquierdo
{1.5cm}                        % margen superior
{16.5cm}                      % anchura del texto
{23.42cm}                    % altura del texto
{10pt}                           % altura de los encabezados
{1cm}                           % espacio entre el texto y los encabezados
{0pt}                             % altura del pie de página
{2cm}     

\lstset{
backgroundcolor=\color{lbcolor},
    tabsize=4,    
%   rulecolor=,
    language=[GNU]C++,
        basicstyle=\tiny,
        aboveskip={1.5\baselineskip},
        columns=fixed,
        showstringspaces=false,
        extendedchars=false,
        breaklines=true,
        prebreak = \raisebox{0ex}[0ex][0ex]{\ensuremath{\hookleftarrow}},
        frame=single,
        numbers=left,                    
	numbersep=5pt, 
        showtabs=false,
        showspaces=false,
        showstringspaces=false,
        identifierstyle=\ttfamily,
        keywordstyle=\color[rgb]{0,0,1},
        commentstyle=\color[rgb]{0.026,0.112,0.095},
        stringstyle=\color{red},
        numberstyle=\color[rgb]{0.205, 0.142, 0.73},
%        \lstdefinestyle{C++}{language=C++,style=numbers}’.
}

\begin{document}

\section{Problemas}
\section{Getting Started}
  \subsection{Insert-Sort}
    \subsubsection{} Using Figure 2.2 as a model, illustrate the operation of INSERTION-SORT on the
    array \hspace{3mm}  A = \{31,41,59,26,41,58\} \\
    \begin{itemize}
     \item 31,\textbf{41},59,26,41,68   (The key is 41. This remains in place).
     \item 31,41,\textbf{59},26,41,68   (The key is 59. This remains in place).
     \item 31,41,59,\textbf{26},41,58   (The key is 26. This takes the position of the number 31).
     \item 26,31,41,49,\textbf{41},58   (The key is 41. This takes the position of the number 59).
     \item 26,31,41,41,59,\textbf{58}   (The key is 58. This takes the position of the number 59).
     \item \textbf{Result: } 26,31,41,41,58,59
    \end{itemize}
    \subsubsection{} Rewrite the INSERTION-SORT procedure to sort into nonincreasing instead of 
    nondecreasing order. \\
    \begin{lstlisting}
void insertSort(vector<int> &vec){
	for(int j = 1; j < vec.size(); j++){
		int key = vec[j];
		int i = j - 1;
		while(i >= 0 and vec[i] < key){
			swap(vec[i + 1], vec[i]);
			i--;
		}
	}
}
    \end{lstlisting}

    \subsubsection{} Consider the \textbf{searching problem:} \\
    \textbf{Input: } A sequence of \textit{i} such that \textit{v}
    = A[i] or the special value \textit{NIL} if \textit{v} does not appear in A.\\
    Write pseudocode for \textbf{linear search}, which scans through the sequence, looking
    for \textit{v}. Using a loop invariant, prove that your algorithm is correct. Make sure that
    your loop invariant fulfills the three necessary properties.\\
    
    \begin{lstlisting}
for (int j = 0; i < A.size(); j++{
  if(v == A[j]) return j;
}
return NULL;
    \end{lstlisting}
    
    \subsubsection{} Cosider the problem of adding two \textit{n-bit}
    binary integers, stored in two \textit{n-element} arrays A and B. The sum
     of the two integers should be stored in binaty form in an (n+1)-elemnt array C.
     State the problem formally and write pseudocode for adding the two integers.
     
     \begin{lstlisting}
vector<int> sumBin(vector<int> &A, vector<int> &B){
	int n = A.size();
	vector<int> C(n + 1);
	int l = 0;
	for(int i = n - 1; i >= 0; i--){
		if(!A[i] and !B[i]){
			C[i + 1]  = l;
			l = 0; 
		}
		else if(A[i] and B[i]){
			C[i + 1] = l;
			if(!l) l = !l;
		}
		else{
			if((A[i] or B[i]) and l){
				C[i + 1] = 0;
				l = 1;
			}
			else C[i + 1] = 1;
		}
	}
	C[0] = l;
	return C;
}
     \end{lstlisting}

  \subsection{Analyzing algorithms}
    \subsubsection{} Express the function $n^3/1000-100n^2+3$ in
   terms of $\Theta$-notation \\
   
  $$n^3/1000 - 100n^2 - 100n + 3 < n^3 - 100n^3 - 100n^3 + 3n^3$$
  $$n^3/1000 - 100n^2 - 100n + 3 < 203n^3$$
  $$\Theta(n^3)$$
  
    \subsubsection{} Consider sorting $n$ numbers stored in array $A$
    by first finding the smallest element of $A$ and exchanging it with
    the element in $A[1]$. The find the second smallest element of $a$
    , and exchange it with $A[2]$. Continue in this manner for the first $n-1$
    element of $A$. Write pseudocode for this algorithm, which is know as 
    \textbf{selection sort}.
    
    \begin{itemize}
     \item  \textbf {What lop invariant does this algorithm mainthain?}
     
     \item \textbf{ Why does it need to run only the first $n-1$ rather than for all $n$ elements?}  
	  \hspace{2mm} Because allways the last element is in the correct site.
     
     \item \textbf{Give the best-case and worst-case running times of
     selection sort int $\Theta$-notation.}
     
     \begin{tabular}{l l l}
      SELECTION SORT & cost & times \\
      1	$n$ = A.length & $C1$ & $1$ \\
      2 for $i = 0$ to $n-1$ & $C2$ & $n$ \\
      3	\hspace{0.5cm} $menor =$ NUMERO INFINITAMENTE GRANDE & $C3$ & $n-1$ \\
      4 \hspace{0.5cm} $index = 0$ & $C4$ & $n-1$ \\
      5	\hspace{0.5cm} for $j = i$ to $n$ & $C5$ & $\sum_{i=1}^{n+1} i$ \\
      6 \hspace{1cm} if $A[j] <  menor$ then & $C6$ & $\sum_{i=0}^{n} i$ \\
      7 \hspace{1.5cm} $menor = A[j]$ & $C7$ & $\sum_{i=0}^{n} i$ \\
      8 \hspace{1.5cm} $index = j$ & $C8$ & $\sum_{i=0}^{n} i$ \\
      9 \hspace{0.5cm} swap($A[index]$,$A[i]$) & $C9$ & $n-1$	
     \end{tabular}

     
     
      $ T(n) = C1 + nC2 + C3(n-1) + C4(n-1) + C5(\sum_{i=1}^{n+1} i) \\     
	+  C6(\sum_{i=0}^{n} i) + C7(\sum_{i=0}^{n} i) + C8(\sum_{i=0}^{n} i) + C9(n-1)	\\
    $
	
	$T(n) = C1 + nC2 + nC3 + nC4 + nC9 - C3 - C4 - C9 + C5(\frac{n^2+3n+2}{2}) \\
         +  C6(\frac{n^2+n}{2}) + C7(\frac{n^2+n}{2}) + C8(\frac{n^2+n}{2})   
    $
     
     $ T(n) = n^2(\frac{C6+C7+C8+C5}{2}) + n(C2+C3+C4+C9 + \frac{3C5+C6+C7+C8}{2}) \\
       + C1 - C3 - C4 - C9$
  
     
    \end{itemize}

    
    
    



\end{document}
