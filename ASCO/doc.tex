\documentclass[a4paper,12pt]{article}
\usepackage[utf8]{inputenc}
\usepackage[spanish]{babel}
\usepackage{color}
\usepackage{parskip}
\usepackage{graphicx}
\usepackage{multirow}
\usepackage{listings}
\usepackage{vmargin}
\usepackage{datetime}
\newdate{date}{31}{09}{2017}
\graphicspath{ {imagenes/} }
\definecolor{mygreen}{rgb}{0,0.6,0}
\definecolor{lbcolor}{rgb}{0.9,0.9,0.9}
\usepackage{epstopdf}
\usepackage{float}


\setpapersize{A4}
\setmargins{2.5cm}       % margen izquierdo
{1.5cm}                        % margen superior
{16.5cm}                      % anchura del texto
{23.42cm}                    % altura del texto
{10pt}                           % altura de los encabezados
{1cm}                           % espacio entre el texto y los encabezados
{0pt}                             % altura del pie de página
{2cm}     

\lstset{
backgroundcolor=\color{lbcolor},
    tabsize=4,    
%   rulecolor=,
    language=[GNU]C++,
        basicstyle=\tiny,
        aboveskip={1.5\baselineskip},
        columns=fixed,
        showstringspaces=false,
        extendedchars=false,
        breaklines=true,
        prebreak = \raisebox{0ex}[0ex][0ex]{\ensuremath{\hookleftarrow}},
        frame=single,
        showtabs=false,
        showspaces=false,
        showstringspaces=false,
        identifierstyle=\ttfamily,
        keywordstyle=\color[rgb]{0,0,1},
        commentstyle=\color[rgb]{0.026,0.112,0.095},
        stringstyle=\color{red},
        numberstyle=\color[rgb]{0.205, 0.142, 0.73},
%        \lstdefinestyle{C++}{language=C++,style=numbers}’.
}


\begin{document}
\title{Análisis del Video}
\author{
Christofer Fabián Chávez Carazas \\
\small{Universidad Nacional de San Agustín de Arequipa} \\
\small{Escuela Profesional de Ciencia de la Computación} \\
\small{Aspectos Sociales y Profesionales de la Ciencia de la Computación}
}
\date{\displaydate{date}}

\maketitle

\begin{itemize}
 \item \textbf{Tiempo Atemporal}
 
 Una de las características que se puede ver en todo el video son la cantidad de historias de fondo: dos esgrimistas peleando, una gimnasta haciendo una presentación, una vaquera con
 su toro mecánico, varios animales, estatuas humanas, y muchas más. Pero lo que nos interesa es que todas las historias no tienen que ver ninguna con la otra, y todas las
 historias aparecen con mucho desorden durante todo el video. Esta es una de las características del tiempo atemporal. Otra característica más que se puede notar en el
 video es que muchas de las imágenes pasan muy rápido y, al estar desordenadas, muchas pasan desapercibidas. A esta característica se la llama compresión del tiempo.
 En conclusión, se comprime el tiempo y se ponen las escenas muy desordenadas para abarcar más historias y, en consecuencia, más mensaje para la canción.
 
 \item \textbf{Espacio de flujos}
 
 Si nos alejamos un poco del contenido del video, y nos acercamos más a lo que es en sí el video, YouTube es un gran espacio de flujos, ya que allí no sólo se
 puede escuchar música, sino ver diferente contenido de diferentes partes del mundo. En el caso de este video, se utiliza para comunicar todos los sentimientos que contienen
 las diferentes escenas complementado con la letra, además de todos los mensajes ``ocultos'' que tiene el video y que se mencionarán más adelante. 
 
 
 \item \textbf{Sujeto posmoderno}
 
 La canción, junto con las imágenes del video, nos habla mucho del amor, el deseo, la lujuria y la ira que se desencadena por las anteriores. Una característica de
 la sociedad actual es que al Sujeto posmoderno le importa mucho sus sentimientos, en el sentido de que le gusta expresarlos; no sólo los sentimientos de amor y felicidad,
 sino también los de ira y odio. También, en el video se muestra una faceta voyeuristica del cuerpo de la mujer, en especial las escenas de la vaquera con el toro mecánico,
 que están hechas para que el centro de atención sea el cuerpo de la vaquera.
 
 \newpage
 
 \item \textbf{Análisis General}
 
 La canción también presenta características de la Sociedad posmoderna en general. En las escenas de la vaquera, el toro mecánico representa al hombre que es controlado por
 la mujer. Sólo en la sociedad posmoderna se está viendo que la mujer tiene más poder en diferentes ámbitos, y que ya no necesariamente el hombre es el sustento o el jefe de la familia.
 En las escenas en donde dos grupos se lanzan pintura, se hace referencia al uso de la pintura en vez de las balas. Hay un llamado a hacer el arte en lugar de la guerra:
 ``Make art, not war'' que es una forma postmoderna de decir ``Make love, not war''  En conclusión, se podría decir que el video es una representación posmodernista del
 arte y la música y el erotismo.
 
 
 
\end{itemize}


\end{document}

