\documentclass[a4paper,12pt]{article}
\usepackage[utf8]{inputenc}
\usepackage[spanish]{babel}
\usepackage{color}
\usepackage{parskip}
\usepackage{graphicx}
\usepackage{multirow}
\usepackage{listings}
\usepackage{vmargin}
\graphicspath{ {imagenes/} }
\definecolor{mygreen}{rgb}{0,0.6,0}
\definecolor{lbcolor}{rgb}{0.9,0.9,0.9}
\usepackage{epstopdf}


\setpapersize{A4}
\setmargins{2.5cm}       % margen izquierdo
{1.5cm}                        % margen superior
{16.5cm}                      % anchura del texto
{23.42cm}                    % altura del texto
{10pt}                           % altura de los encabezados
{1cm}                           % espacio entre el texto y los encabezados
{0pt}                             % altura del pie de página
{2cm}     

\lstset{
backgroundcolor=\color{lbcolor},
    tabsize=4,    
%   rulecolor=,
    language=[GNU]C++,
        basicstyle=\tiny,
        aboveskip={1.5\baselineskip},
        columns=fixed,
        showstringspaces=false,
        extendedchars=false,
        breaklines=true,
        prebreak = \raisebox{0ex}[0ex][0ex]{\ensuremath{\hookleftarrow}},
        frame=single,
        showtabs=false,
        showspaces=false,
        showstringspaces=false,
        identifierstyle=\ttfamily,
        keywordstyle=\color[rgb]{0,0,1},
        commentstyle=\color[rgb]{0.026,0.112,0.095},
        stringstyle=\color{red},
        numberstyle=\color[rgb]{0.205, 0.142, 0.73},
%        \lstdefinestyle{C++}{language=C++,style=numbers}’.
}

\begin{document}

\begin{LARGE}
 CHRISTOFER FABIAN CHÁVEZ CARAZAS
\end{LARGE}

\section{Diga qué es un índice primario y secundario. Ponga un ejemplo}

  \subsection{Índices primarios}
  
    Se basan principalmente en archivos ordenados secuencialmente.
    Se denominan índice primario cuando el archivo de datos asociado 
    se encuentra ordenado en base a la llave de búsqueda. \\
    
    \textbf{Ejemplo:}El ejemplo más utilizado vendría ser el 
    de la biblioteca que se ordena por nombre de autor o de libro.
    
    
  \subsection{Índices secundarios}
  
    Se basan principalmente en archivos \textbf{NO} ordenados.
    Mantienen una organización externa asociado a otros datos.
    Permite hacer referencia a una misma estructura.
    Un índice mantiene la llave de búsqueda y el otro índice la organicación del archivo. \\
    
    \textbf{Ejemplo:}Un ejemplo podría ser una tabla con el saldo de las personas.
    El índice secundario de dicha tabla puede estar con la clave saldo.
    
\section{¿Cuál es el efecto que logra el atributo o la característica
	  clustered?}
	  En un índice non-clustered, la clave por la que buscamos tiene un puntero a la página de datos donde se encuentra el registro. Mientras que en índice clustered, la leaf level es la pagina de datos!. Con lo cual, el gestor de base de datos, se ahorra hacer un salto para leer los datos del registro (Bookmark lookup)
    
    \newpage
    
\section{Cómo guarda la tabla y el índice en el siguiente ejemplo:}


  \begin{itemize}
   \item Create clustered index IDX.dinero on Tabla(Dinero)
   \item Create index IDX.dinero on Tabla(Dinero)
  \end{itemize}

  Con el primero la tabla se ordena físicamente segun el índice
  Dinero.
  Con el segundo se crea un index secundario con el atributo Dinero.
  
  
  
    

    
    
\end{document}
