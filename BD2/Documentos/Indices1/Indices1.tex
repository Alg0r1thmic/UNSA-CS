\documentclass[a4paper,12pt]{article}
\usepackage[utf8]{inputenc}
\usepackage[spanish]{babel}
\usepackage{color}
\usepackage{parskip}
\usepackage{graphicx}
\usepackage{multirow}
\usepackage{listings}
\usepackage{vmargin}
\graphicspath{ {imagenes/} }
\definecolor{mygreen}{rgb}{0,0.6,0}
\definecolor{lbcolor}{rgb}{0.9,0.9,0.9}
\usepackage{epstopdf}


\setpapersize{A4}
\setmargins{2.5cm}       % margen izquierdo
{1.5cm}                        % margen superior
{16.5cm}                      % anchura del texto
{23.42cm}                    % altura del texto
{10pt}                           % altura de los encabezados
{1cm}                           % espacio entre el texto y los encabezados
{0pt}                             % altura del pie de página
{2cm}     

\lstset{
backgroundcolor=\color{lbcolor},
    tabsize=4,    
%   rulecolor=,
    language=[GNU]C++,
        basicstyle=\tiny,
        aboveskip={1.5\baselineskip},
        columns=fixed,
        showstringspaces=false,
        extendedchars=false,
        breaklines=true,
        prebreak = \raisebox{0ex}[0ex][0ex]{\ensuremath{\hookleftarrow}},
        frame=single,
        showtabs=false,
        showspaces=false,
        showstringspaces=false,
        identifierstyle=\ttfamily,
        keywordstyle=\color[rgb]{0,0,1},
        commentstyle=\color[rgb]{0.026,0.112,0.095},
        stringstyle=\color{red},
        numberstyle=\color[rgb]{0.205, 0.142, 0.73},
%        \lstdefinestyle{C++}{language=C++,style=numbers}’.
}

\begin{document}

\begin{LARGE}
 CHRISTOFER FABIAN CHÁVEZ CARAZAS
\end{LARGE}

\section{Diga qué es un índice primeario y secundario. Ponga un ejemplo}

  \subsection{Índices primarios}
  
    Se basan principalmente en archivos ordenados secuencialmente.
    Se denominan índice primario cuando el archivo de datos asociado 
    se encuentra ordenado en base a la llave de búsqueda. \\
    
    \textbf{Ejemplo->}El ejemplo más utilizado vendría ser el 
    de la biblioteca que se ordena por nombre de autor o de libro.
    
    
  \subsection{Índices secundarios}
  
    Se basan principalmente en archivos \textbf{NO} ordenados.
    Mantienen una organización externa asociado a otros datos.
    Permite hacer referencia a una misma estructura.
    Un índice mantiene la llave de búsqueda y el otro índice la organicación del archivo. \\
    
    \textbf{Ejemplo->}Un ejemplo 
    

    
    
\end{document}
