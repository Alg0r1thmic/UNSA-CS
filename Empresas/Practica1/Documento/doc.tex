\documentclass[a4paper,12pt]{article}
\usepackage[utf8]{inputenc}
\usepackage[spanish]{babel}
\usepackage{color}
\usepackage{parskip}
\usepackage{graphicx}
\usepackage{multirow}
\usepackage{listings}
\usepackage{vmargin}
\graphicspath{ {imagenes/} }
\definecolor{mygreen}{rgb}{0,0.6,0}
\definecolor{lbcolor}{rgb}{0.9,0.9,0.9}
\usepackage{epstopdf}


\setpapersize{A4}
\setmargins{2.5cm}       % margen izquierdo
{1.5cm}                        % margen superior
{16.5cm}                      % anchura del texto
{23.42cm}                    % altura del texto
{10pt}                           % altura de los encabezados
{1cm}                           % espacio entre el texto y los encabezados
{0pt}                             % altura del pie de página
{2cm}     



\begin{document}
\title{Práctica de Empresas}
  \author{
  Christofer Fabián Chávez Carazas \\
  \small{Universidad Nacional de San Agustín} \\
  \small{Empresas I}
}

\maketitle


\begin{itemize}
 \item \textbf{¿Cuál es la región más innovadora del mundo?}
 
 
 \item \textbf{¿Áreas prioritarias en Ciencia y Tecnología en el Perú?}
 
 
 \item \textbf{¿Cuáles son los motivos para plantearse la creación de una empresa?}
 
 
 \item \textbf{¿Qué elementos principales conforman una empresa?}
 
 
 \item \textbf{¿Qué características posee una empresa de Base Tecnológica?}
 
 
 \item \textbf{¿Tiene alguna idea de negocio a emprender?¿Cuál?}
 
 
 \item \textbf{¿Tiene alguna experiencia como empresario o gerente?}
\end{itemize}



\end{document}

