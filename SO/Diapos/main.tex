\documentclass{beamer}
\usepackage[spanish]{babel}
\usepackage[utf8]{inputenc}

\usetheme{Madrid}


\title{Principios del Hardware de E/S}

\author{Christofer Chávez Carazas
        \and Karen Gordillo Viña
        \and Ruben Torres Lima}


\institute[Universities of Somewhere and Elsewhere] 
{Universidad Nacional de San Agustín \\ Ciencia de la Computación}


\date{\today}




\AtBeginSubsection[]
{
  \begin{frame}<beamer>{Outline}
    \tableofcontents[currentsection,currentsubsection]
  \end{frame}
}


\begin{document}

\begin{frame}
  \titlepage
\end{frame}

%\begin{frame}{Contenidos}
 % \tableofcontents
%\end{frame}



\begin{frame}{Introducción}
\begin{itemize}
    

 \item  Una de las funciones principales de un sistema operativo es el control de todos los dispositivos de entrada/salida de la computadora.
  
  \item Debe enviar comandos a los dispositivos, detectar las interrupciones y controlar los errores.
  \item Debe proporcionar además una interfaz entre los dispositivos y el resto del sistema, sencilla y fácil de usar.
\end{itemize}
\end{frame}

\begin{frame}{Dispositivos de E/S}
    \begin{block}{}
    Dispositivos de bloque
    \end{block}
    \begin{block}{}
    Dispositivos de carácter
    \end{block}
\end{frame}

\begin{frame}{Dispositivos de bloque}
    \begin{itemize}
    
    \item Un dispositivo de bloque es aquel que almacena la información en bloques de tamaño fijo, cada uno con su propia dirección.
    
    
    \item La propiedad esencial de un dispositivo de bloque es la posibilidad de leer o escribir en un bloque de forma independiente de los demás.
    
    
    \item Los tamaños comunes de los bloques van desde 128 bytes hasta 1024 bytes.
    
    
    \item Los discos son dispositivos de bloque.
    
    \end{itemize}
    
\end{frame}

\begin{frame}{Dispositivos de carácter}
    \begin{itemize}
        
    \item Un dispositivo de caracter envía o recibe un flujo de caracteres, sin sujetarse a una estructura de bloques.
    
    \item Las terminales, impresoras de la línea, cintas de papel, tarjetas perforadas, interfaces de una red, ratones (para apuntar hacia la pantalla), y muchos otros dispositivos no parecidos a los discos son dispositivos de caracter.
    
    \end{itemize}
\end{frame}


\begin{frame}{Controladores de Dispositivos}
  \begin{block}{Las unidades de E/S generalmente constan de:}
    \begin{itemize}
      \item Un componente mecánico.
      \item Un componente electrónico, el controlador del dispositivo o adaptador.  
    \end{itemize}
  \end{block}
  
  \begin{block}{Los modelos más frecuentes de comunicación
entre la cpu y los controladores son:}

    \begin{itemize}
        \item Para la mayoría de las micro y mini computadoras se
        utiliza el modelo de bus del sistema.
        \item Para la mayoría de los mainframes se utiliza el modelo
        de varios buses y computadoras especializadas en E/S.
        llamadas canales de E/S.
    \end{itemize}

  \end{block}
\end{frame}

\begin{frame}{Controladores de Dispositivos}
    \begin{block}{La comunicación es mediante un flujo de bits en serie que:}
    \begin{itemize}
        \item Comienza con un preámbulo.
        \item Sigue con una serie de bits (de un sector de disco, por ej.).
        \item Concluye con una suma para verificación o un código corrector de errores.
    \end{itemize}
    \end{block}
    
    \begin{block}{El preámbulo:}
    \begin{itemize}
        \item Se escribe al dar formato al disco.
        \item Contiene el número de cilindro y sector, el tamaño de sector y otros datos similares.
    \end{itemize}
    \end{block}
\end{frame}

\begin{frame}{Controladores de Dispositivos}
    \begin{block}{El controlador debe:}
    \begin{itemize}
        \item Convertir el flujo de bits en serie en un bloque de bytes.
        \item Efectuar cualquier corrección de errores necesaria.
        \item Copiar el bloque en la memoria principal.
    \end{itemize}
    \end{block}
    
    \begin{block}{Cada controlador posee registros que utiliza para comunicarse con la cpu:}
    \begin{itemize}
    \item Pueden ser parte del espacio normal de direcciones de la memoria: E/S mapeada a memoria.
    \item Pueden utilizar un espacio de direcciones especial para la E/S, asignando a cada controlador una parte de él.
    \end{itemize}

    \end{block}
    
\end{frame}


\begin{frame}{Controladores de Dispositivos}
  \begin{block}{}
  El S.O. realiza la E/S al escribir comandos en los registros de los controladores; los parámetros de los comandos también se cargan en los registros de los controladores. Al aceptar el comando, la cpu puede dejar al controlador y dedicarse a otro trabajo.
  \end{block}
  
  \begin{block}{Al terminar el comando, el controlador provoca una interrupción para permitir que el S.O.:}
  \begin{itemize}
    \item Obtenga el control de la cpu.
    \item Verifique los resultados de la operación.
\end{itemize}
  \end{block}
  \begin{block}{}
  La cpu obtiene los resultados y el estado del dispositivo
al leer uno o más bytes de información de los registros
del controlador.
  \end{block}
\end{frame}

\begin{frame}{Acceso Directo a Memoria (DMA)}
    \begin{block}{}
        El acceso directo a memoria (DMA) permite a cierto tipo de componentes de una computadora acceder a la memoria del sistema para leer o escribir independientemente de la unidad central de procesamiento (CPU) principal.
    \end{block}
\end{frame}

\begin{frame}{Acceso Directo a Memoria (DMA)}
    \begin{block}{}
        Lectura sin DMA
    \end{block}
    \begin{block}{}
        Lectura con DMA
    \end{block}
\end{frame}

\begin{frame}{Lectura sin DMA}
        \begin{block}{
            El controlador lee en serie el bloque (uno o más sectores) de la unidad:
            \pause 
        }
        \begin{itemize}
            \item {
                La lectura es bit por bit.
                \pause
            }
            \item {
                Los bits del bloque se graban en el buffer interno del controlador.
                \pause
            }
        \end{itemize}
        
        \end{block}
        
        \begin{block}{
            Se calcula la suma de verificación para corroborar que no existen errores de lectura.
            
        }
        \end{block}
        \pause
        \begin{block}{
            El controlador provoca una interrupción.
        }
        \end{block}
        
\end{frame}

\begin{frame}{Lectura sin DMA}
        \begin{block}{
            El S. O. lee el bloque del disco por medio del buffer del controlador:
            \pause 
        }
        \begin{itemize}
            \item {
                La lectura es por byte o palabra a la vez.
                \pause
            }
            \item {
                En cada iteración de este ciclo se lee un byte o una palabra del registro del controlador y se almacena en memoria.
                \pause
            }
        \end{itemize}
        \end{block}
        
        \begin{block}{
            Se desperdicia tiempo de la CPU.
        }
        \end{block}
        
\end{frame}

\begin{frame}{Lectura con DMA}
        \begin{block}{
            La CPU proporciona al controlador:
            \pause 
        }
        \begin{itemize}
            \item {
                La dirección del bloque en el disco.
                \pause
            }
            \item {
                La dirección en memoria adonde debe ir el bloque.
                \pause
            }
            \item {
                 El número de bytes por transferir.
            }
        \end{itemize}
        \end{block}
        
        
\end{frame}

\begin{frame}{Lectura con DMA}
    
        \begin{block}{
            Luego de que el controlador leyó todo el bloque del dispositivo a su buffer y de que corroboró la suma de verificación:
            \pause
        }
        \begin{itemize}
            \item {
                Copia el primer byte o palabra a la memoria principal.
                \pause
            }
            \item {
                 Lo hace en la dirección especificada por medio de la dirección de memoria de DMA.
                \pause
            }
            \item {
                 Incrementa la dirección DMA y decrementa el contador DMA en el número de bytes que acaba de transferir.
                \pause
            }
            \item {
                Se repite este proceso hasta que el contador se anula y por lo tanto el controlador provoca una interrupción.
                \pause
            }
            \item {
                 Al iniciar su ejecución el S.O. luego de la interrupción provocada, no debe copiar el bloque en la memoria, porque ya se encuentra ahí.
            }
            
        \end{itemize}
        \end{block}
        
\end{frame}

\begin{frame}{Lectura con DMA}
    
        \begin{block}{
            El controlador necesita un buffer interno porque una vez iniciada una transferencia del disco:
            \pause
        }
        \begin{itemize}
            \item {
                Los bits siguen llegando del disco constantemente.
                \pause
            }
            
            \item {
                 No interesa si el controlador está listo o no para recibirlos.
                \pause
            }
            \item {
                 Si el controlador intentara escribir los datos en la memoria directamente:
                \pause
                \begin{itemize}
                    \item {
                        Tendría que recurrir al bus del sistema para c / u de las palabras (o bytes) transferidas.
                        \pause
                    }
                    \item {
                        El bus podría estar ocupado por otro dispositivo y el controlador debería esperar.
                        \pause
                    }
                    \ item {
                        Si la siguiente palabra llegara antes de que la anterior hubiera sido almacenada, el controlador la tendría que almacenar en alguna parte.
                        \pause
                    }
                \end{itemize}
            }
            
        \end{itemize}
        \end{block}
        
\end{frame}

\begin{frame}{Lectura con DMA}
    
        \begin{block}{
            Si el bloque se guarda en un buffer interno:
            \pause
        }
        \begin{itemize}
            \item {
                El bus no se necesita sino hasta que el DMA comienza.
                \pause
            }
            
            \item {
                 La transferencia DMA a la memoria ya no es un aspecto crítico del tiempo.
            }
            
        \end{itemize}
        \end{block}
        
\end{frame}


\begin{frame}{Lectura con DMA}
    
        \begin{block}{
            Los controladores simples no pueden atender la E/S simultánea:
            \pause
        }
        \begin{itemize}
            \item {
                Mientras transfieren a la memoria, el sector que pasa debajo de la cabeza del disco se pierde; es decir que el bloque siguiente al recién leído se pierde.
                \pause
            }
            
            \item {
                 La lectura de una pista completa se hará en dos rotaciones completas, una para los bloques pares y otra para los impares.
                
            }

            
        \end{itemize}
        \end{block}
        
\end{frame}

\begin{frame}{Lectura con DMA}
    
        \begin{block}{
            Los controladores simples no pueden atender la E/S simultánea:
            \pause
        }
        \begin{itemize}
            
            \item {
                 Si el tiempo necesario para una transferencia de un bloque del controlador a la memoria por medio del bus es mayor que el tiempo necesario para leer un bloque del disco:
                \pause
                \begin{itemize}
                    \item {
                        Sería necesario leer un bloque y luego saltar dos o más bloques.
                        \pause
                    }
                    \item {
                        El salto de bloques:
                        \pause
                        \begin{itemize}
                            \item {
                                Se ejecuta para darle tiempo al controlador para la transferencia de los datos a la memoria.
                                \pause
                            }
                            \item {
                                Se llama separación.
                                \pause
                            }
                            \item{
                                Al formatear el disco, los bloques se numeran tomando en cuenta el factor de separación.
                                \pause
                            }
                        \end{itemize}
                    }
                    \item {
                        El bus podría estar ocupado por otro dispositivo y el controlador debería esperar.
                        \pause
                    }
                    \item {Esto permite al S. O.: 
                        \pause
                    }
                    \begin{itemize}
                        \item {
                            Leer los bloques con numeración consecutiva.
                            \pause
                        }
                        \item {
                            Conservar la máxima velocidad posible del hardware.
        
                        }
                    \end{itemize}
                \end{itemize}
            }
            
        \end{itemize}
        \end{block}
        
\end{frame}

\begin{frame}{Conclusiones}
\begin{itemize}
    \item A pesar de que los dispositivos de hardware se encuentran clasificados, no todos los dispositivos encajan perfectamente en dichas clasificaciones.
    \item El trabajo de lectura sin DMA muestra desperdicio de tiempo en la CPU.
    \item Con la disposicion de un buffer interno el tiempo ya no es critico en la transferencia de la DMA a la memoria.
    \item Los controladores simples no tienen capacidad de atencion simultanea por lo cual se generan perdidas de bloques mientras otros se tranfieren a memoria.
\end{itemize}
\end{frame}

\end{document}


