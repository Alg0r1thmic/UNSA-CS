\documentclass[a4paper,12pt]{article}
\usepackage[utf8]{inputenc}
\usepackage[spanish]{babel}
\usepackage{color}
\usepackage{parskip}
\usepackage{graphicx}
\usepackage{multirow}
\usepackage{listings}
\definecolor{mygreen}{rgb}{0,0.6,0}
\definecolor{lbcolor}{rgb}{0.9,0.9,0.9}
\DeclareGraphicsExtensions{.pdf,.png,.eps}
\usepackage{epstopdf}

\lstset{
backgroundcolor=\color{lbcolor},
    tabsize=4,    
%   rulecolor=,
    language=SQL,
        basicstyle=\scriptsize,
        aboveskip={1.5\baselineskip},
        columns=fixed,
        showstringspaces=false,
        extendedchars=false,
        breaklines=true,
        prebreak = \raisebox{0ex}[0ex][0ex]{\ensuremath{\hookleftarrow}},
        frame=single,
        numbers=left,
        showtabs=false,
        showspaces=false,
        showstringspaces=false,
        identifierstyle=\ttfamily,
        keywordstyle=\color[rgb]{0,0,1},
        commentstyle=\color[rgb]{0.026,0.112,0.095},
        stringstyle=\color{red},
        numberstyle=\color[rgb]{0.205, 0.142, 0.73},
%        \lstdefinestyle{C++}{language=C++,style=numbers}’.
}

\begin{document}

\begin{Large}
\textbf{Hecho por:} Christofer Fabián Chávez Carazas
\end{Large}

\section{Creando base de datos con sus respectivas tablas}

\begin{lstlisting}
create database sesion3;
use sesion3;

create table Clientes(
NCliente int not null,
Nombre varchar(100) not null,
Dir varchar(200),
Telefono char(8),
Poblacion varchar(100) not null);

alter table Clientes
add constraint pk_Clientes
primary key(NCliente);

create table Producto(
CodProducto int not null,
Descripcion varchar(200) not null,
Precio int not null);

alter table Producto
add constraint pk_Producto
primary key(CodProducto);

create table Venta(
IdVenta int not null,
CodProducto int not null,
NCliente int not null,
Cantidad int not null);

alter table Venta
add constraint pk_Venta
primary key(IdVenta);

alter table Venta
add constraint fk_VentaProducto
foreign key(CodProducto)
references Producto(CodProducto);

alter table Venta
add constraint fk_VentaCliente
foreign key(NCliente)
references Clientes(NCliente);

\end{lstlisting}

\section{Insertando filas}

\begin{lstlisting}
use sesion3;

insert into Clientes values(1,'Chris','Huascar','4534354','Arequipa');
insert into Clientes values(2,'Carlos','Dolores','1233423','Lima');
insert into Clientes values(3, 'Julitus', 'Av Maria', '3124324', 'Arequipa');
insert into Clientes values(4, 'Juan', 'C Tacna', '21321321', 'Lima');
insert into Clientes values(5, 'Nicoll', 'Bustamante', '213321', 'Arequipa');

insert into Producto values(1,'Papel', 100);
insert into Producto values(2, 'Ropa', 200);
insert into Producto values(3, 'Comida', 300);
insert into Producto values(4, 'TV', 500);
insert into Producto values(5, 'Compu', 1000);

insert into Venta values(1,5,1,3);
insert into Venta values(2,1,5,100);
insert into Venta values(3,3,2,300);
insert into Venta values(4,3,5,500);
insert into Venta values(5,4,3,50);

insert into Venta values(18,1,1,30);
\end{lstlisting}

\section{Consultas}

\begin{enumerate}

\item Nombre de los Clientes que viven en Arequipa.

\begin{lstlisting}
select Nombre from clientes where Poblacion like 'Arequipa';
\end{lstlisting}

\item Código de Producto y Descripción de los productos cuyo código y descripción coincidan.

\begin{lstlisting}
select CodProducto, Descripcion from Producto where CodProducto = Descripcion;
\end{lstlisting}


\end{enumerate}

\end{document}