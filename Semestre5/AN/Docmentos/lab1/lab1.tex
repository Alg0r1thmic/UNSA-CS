\documentclass[a4paper,12pt]{article}
\usepackage[utf8]{inputenc}
\usepackage[spanish]{babel}
\usepackage{color}
\usepackage{parskip}
\usepackage{graphicx}
\usepackage{multirow}
\usepackage{listings}
\usepackage{vmargin}
\graphicspath{ {imagenes/} }
\definecolor{mygreen}{rgb}{0,0.6,0}
\definecolor{lbcolor}{rgb}{0.9,0.9,0.9}
\usepackage{epstopdf}


\setpapersize{A4}
\setmargins{2.5cm}       % margen izquierdo
{1.5cm}                        % margen superior
{16.5cm}                      % anchura del texto
{23.42cm}                    % altura del texto
{10pt}                           % altura de los encabezados
{1cm}                           % espacio entre el texto y los encabezados
{0pt}                             % altura del pie de página
{2cm}     

\lstset{
backgroundcolor=\color{lbcolor},
    tabsize=4,    
%   rulecolor=,
    language=[GNU]C++,
        basicstyle=\tiny,
        aboveskip={1.5\baselineskip},
        columns=fixed,
        showstringspaces=false,
        extendedchars=false,
        breaklines=true,
        prebreak = \raisebox{0ex}[0ex][0ex]{\ensuremath{\hookleftarrow}},
        frame=single,
        showtabs=false,
        showspaces=false,
        showstringspaces=false,
        identifierstyle=\ttfamily,
        keywordstyle=\color[rgb]{0,0,1},
        commentstyle=\color[rgb]{0.026,0.112,0.095},
        stringstyle=\color{red},
        numberstyle=\color[rgb]{0.205, 0.142, 0.73},
%        \lstdefinestyle{C++}{language=C++,style=numbers}’.
}

\begin{document}

\section{Ejerccio 1}

\begin{lstlisting}
#include <iostream>
#include <cmath>

using namespace std;

int cifras_sig(float xs, float x){
	float v = (abs(x - xs)) / abs(x);
	int i = 0;
	for(; v < 0.5 * pow(10,-1*i); i++)
	{
		cout<<v<<"<"<<0.5 * pow(10,-1*i)<<endl;
	}
	return i - 1;
}

int main(){
	float xs;
	float x;
	cout<<"Ingrese xs->"<<endl;
	cin>>xs;
	cout<<"Ingrese x->"<<endl;
	cin>>x;
	cout<<cifras_sig(xs,x)<<endl;
}
\end{lstlisting}

\section{Ejercicio 2}

\begin{lstlisting}
function f()
	x= linspace(1-2*10^-8,1+2*10^-8,450);
	y =  x.^ 7 - 7 * x.^ 6 + 21 * x.^ 5 - 35 * x.^ 4 + 35 * x.^ 3 - 21 * x.^ 2 + 7 * x - 1;
	plot(y);
	grid on
end
\end{lstlisting}

\section{Ejercicio 3}

\begin{lstlisting}
function terminos(n)
	res = [];
	err = [];
	for i = 2:n + 1
		if(i == 2)
			res(i - 1) = 2;
		else
			res(i -1) = (2^(i-1/2)) * sqrt(1- sqrt(1- (4^(1-i)) * (res(i-2)^2)));
		end
		err(i - 1) = abs((pi - res(i-1)) / res(i-1));
	end
	[res' err']
	grid on
	plot(res,err)
end
\end{lstlisting}


\end{document}
