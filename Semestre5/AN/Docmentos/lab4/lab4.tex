\documentclass[a4paper,12pt]{article}
\usepackage[utf8]{inputenc}
\usepackage[spanish]{babel}
\usepackage{color}
\usepackage{parskip}
\usepackage{graphicx}
\usepackage{multirow}
\usepackage{listings}
\usepackage{vmargin}
\graphicspath{ {imagenes/} }
\definecolor{mygreen}{rgb}{0,0.6,0}
\definecolor{lbcolor}{rgb}{0.9,0.9,0.9}
\usepackage{epstopdf}


\setpapersize{A4}
\setmargins{2.5cm}       % margen izquierdo
{1.5cm}                        % margen superior
{16.5cm}                      % anchura del texto
{23.42cm}                    % altura del texto
{10pt}                           % altura de los encabezados
{1cm}                           % espacio entre el texto y los encabezados
{0pt}                             % altura del pie de página
{2cm}     

\lstset{
backgroundcolor=\color{lbcolor},
    tabsize=4,    
%   rulecolor=,
    language=[GNU]C++,
        basicstyle=\tiny,
        aboveskip={1.5\baselineskip},
        columns=fixed,
        showstringspaces=false,
        extendedchars=false,
        breaklines=true,
        prebreak = \raisebox{0ex}[0ex][0ex]{\ensuremath{\hookleftarrow}},
        frame=single,
        showtabs=false,
        showspaces=false,
        showstringspaces=false,
        identifierstyle=\ttfamily,
        keywordstyle=\color[rgb]{0,0,1},
        commentstyle=\color[rgb]{0.026,0.112,0.095},
        stringstyle=\color{red},
        numberstyle=\color[rgb]{0.205, 0.142, 0.73},
%        \lstdefinestyle{C++}{language=C++,style=numbers}’.
}

\begin{document}

\section{OperacionesMatriz.h}

Aquí se definen todas las operaciones de la Matriz que utilizaremos en nuestros métodos.

\begin{lstlisting}
#ifndef OPERACIONESMATRIZ_H
#define OPERACIONESMATRIZ_H

#include <iostream>
#include <cmath>
#include <vector>

using namespace std;

typedef long double Num;
typedef vector<vector<Num>> Matriz;
typedef vector<Num> Lista;

void mostrarLista(Lista &A){
	for(int i = 0; i < A.size(); i++){
		cout<<A[i]<<" ";
	}
	cout<<endl<<endl;;
}

Matriz matrizAumentada(Matriz A, Lista B){
	Matriz res = A;
	for(int i = 0; i < B.size(); i++){
		res[i].push_back(B[i]);
	}
	return res;
}

void mostrarMatriz(Matriz &A){
	for(int i = 0; i < A.size(); i++){
		for(int j = 0; j < A[i].size(); j++){
			cout<<A[i][j]<<" ";
		}
		cout<<endl;
	}
	cout<<endl;
}

int buscarMayor(int ini, int fin, int j, Matriz &A){
	int mayor = abs(A[ini][j]);
	int index = ini;
	for(int i = ini +1; i <= fin; i++){
		if(mayor < abs(A[i][j])){
			mayor = abs(A[i][j]);
			index = i;
		} 
	}
	return index;
}

Lista operator *(Matriz a, Lista b){
	int n = a.size();
	Lista res = b;
	for(int i = 0; i < n; i++){
		for(int k = 0; k < n; k++){
			Num sum = 0;
			for(int j = 0; j < n; j++){
				sum += a[i][j] * b[j];
			}
			res[i] = sum;
		}
	}
	return res;
}

Matriz operator *(Matriz a, Matriz b){
	int n = a.size();
	Matriz res = a;
	for(int i = 0; i < n; i++){
		for(int k = 0; k < n; k++){
			Num sum = 0;
			for(int j = 0; j < n; j++){
				sum += a[i][j] * b[j][k];
			}
			res[i][k] = sum;
		}
	}
	return res;
}

Matriz operator +(Matriz a, Matriz b){
	int n = a.size();
	Matriz res = a;
	for(int i = 0; i < n; i++){
		for(int j = 0; j < n; j++){
			res[i][j] = a[i][j] + b[i][j];
		}
	}
	return res;
}

Matriz zeros(int n){
	Matriz res(n);
	for(int i = 0; i < n; i++){
		res[i] = vector<Num>(n);
	}
	return res;
}

Matriz identidad(int n){
	Matriz res(n);
	for(int i = 0; i < n; i++){
		res[i] = vector<Num>(n);
	}
	for(int i = 0; i < n; i++){
		for(int j = 0; j < n; j++){
			if(i == j) res[i][j] = 1;
		}
	}
	return res;
}

#endif
\end{lstlisting}

\section{Sustituciones.h}

Aquí se definen los dos tipos de sustituciones a usar: La sustitución regresiva y la sustitución progresiva.

\begin{lstlisting}
#ifndef SUSTITUCIONES_H
#define SUSTITUCIONES_H

#include <iostream>
#include <cmath>
#include <algorithm>
#include <vector>
#include "OperacionesMatriz.h"

using namespace std;

Lista SustitucionRegresiva(Matriz A){ //Con Matriz Aumentada
	int j = 0;
	Lista res(A.size());
	for(int i = A.size() - 1; i >= 0; i--, j++){
		Num sum = 0;
		for(int k = 0; k < j; k++){
			sum += A[i][A[i].size() - 2 - k] * res[A[i].size() -2 -k];
		}
		res[A[i].size() -2-j] = (A[i][A[i].size()-1] - sum) / A[i][A[i].size()-2-j];

	}
	return res;
}

Lista SustitucionProgresiva(Matriz A){ //Con Matriz Aumentada
	int j = 0;
	Lista res(A.size());
	for(int i = 0; i < A.size(); i++, j++){
		Num sum = 0;
		for(int k = 0; k < j; k++){
			sum += A[i][k] * res[k];
		}
		res[j] = (A[i][A[i].size()-1] - sum) / A[i][j];
	}
	return res;
}


#endif
\end{lstlisting}

\section{Gauss.cpp}

Aquí está resuelto el método de Gauss con pivoteo.

\begin{lstlisting}
#include <iostream>
#include "Sustituciones.h"
#include "OperacionesMatriz.h"

using namespace std;

Matriz Gauss(Matriz A, Lista &B){
	A = matrizAumentada(A,B);
	int pibot = 0;
	while(pibot != B.size() - 1){
		int mayor = buscarMayor(pibot,B.size()-1,pibot,A);
		if (mayor != pibot) swap(A[pibot],A[mayor]);
		for(int i = pibot+1; i < B.size(); i++){
			Num d = A[i][pibot]/A[pibot][pibot];
			for(int j = pibot; j < A[i].size(); j++){
				A[i][j] = A[i][j] - d*A[pibot][j];
			}
		}
		pibot++;
	}
	return A;
}

int main(){
	Matriz A = {{1,1,-1},{2,-1,1},{4,1,-2}};
	Lista B = {1,2,3};
	Matriz AA = Gauss(A,B);
	Lista res = SustitucionRegresiva(AA);
	for(Num n : res){
		cout<<n<<" ";
	}
	cout<<endl;
}
\end{lstlisting}

\section{PLU.cpp}

Aquí está resuelto el método de PLU con piboteo.

\begin{lstlisting}
#include <iostream>
#include <tuple>
#include "Sustituciones.h"
#include "OperacionesMatriz.h"

using namespace std;

enum Tipes{M_P,M_L,M_U};

tuple<Matriz,Matriz,Matriz> PLU(Matriz A){
	Matriz P = identidad(A.size());
	Matriz L = zeros(A.size());
	Matriz U = A;
	int pibot = 0;
	while(pibot != U.size() - 1){
		int mayor = buscarMayor(pibot,U.size()-1,pibot,U);
		if (mayor != pibot){
			swap(U[pibot],U[mayor]);		
			swap(P[pibot],P[mayor]);
			swap(L[pibot],L[mayor]);
		} 
		for(int i = pibot+1; i < U.size(); i++){
			Num d = U[i][pibot]/U[pibot][pibot];
			L[i][pibot] = d;
			for(int j = pibot; j < U[i].size(); j++){
				U[i][j] = U[i][j] - d*U[pibot][j];
			}
		}
		pibot++;
	}
	L = L + identidad(L.size());
	return make_tuple(P,L,U);
}

int main(){
	Matriz A = {{1,1,-1},{2,-1,1},{4,1,-2}};
	Lista B = {1,2,3};
	auto t = PLU(A);
	Matriz P = get<M_P>(t);
	Matriz L = get<M_L>(t);
	Matriz U = get<M_U>(t);
	mostrarMatriz(P);
	mostrarMatriz(L);
	mostrarMatriz(U);
	Matriz Q = P * A;
	Matriz W = L * U;
	mostrarMatriz(Q);
	mostrarMatriz(W);
	Lista Y = SustitucionProgresiva(matrizAumentada(L,P*B));
	mostrarLista(Y);
	Lista res = SustitucionRegresiva(matrizAumentada(U,Y));
	mostrarLista(res);
}
\end{lstlisting}

\end{document}
