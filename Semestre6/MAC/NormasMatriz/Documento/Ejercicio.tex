\documentclass[a4paper,12pt]{article}
\usepackage[utf8]{inputenc}
\usepackage[spanish]{babel}
\usepackage{color}
\usepackage{parskip}
\usepackage{graphicx}
\usepackage{multirow}
\usepackage{listings}
\usepackage{vmargin}
\graphicspath{ {imagenes/} }
\definecolor{mygreen}{rgb}{0,0.6,0}
\definecolor{lbcolor}{rgb}{0.9,0.9,0.9}
\usepackage{epstopdf}


\setpapersize{A4}
\setmargins{2.5cm}       % margen izquierdo
{1.5cm}                        % margen superior
{16.5cm}                      % anchura del texto
{23.42cm}                    % altura del texto
{10pt}                           % altura de los encabezados
{1cm}                           % espacio entre el texto y los encabezados
{0pt}                             % altura del pie de página
{2cm}     

\lstset{
backgroundcolor=\color{lbcolor},
    tabsize=4,    
%   rulecolor=,
    language=[GNU]C++,
        basicstyle=\tiny,
        aboveskip={1.5\baselineskip},
        columns=fixed,
        showstringspaces=false,
        extendedchars=false,
        breaklines=true,
        prebreak = \raisebox{0ex}[0ex][0ex]{\ensuremath{\hookleftarrow}},
        frame=single,
        showtabs=false,
        showspaces=false,
        showstringspaces=false,
        identifierstyle=\ttfamily,
        keywordstyle=\color[rgb]{0,0,1},
        commentstyle=\color[rgb]{0.026,0.112,0.095},
        stringstyle=\color{red},
        numberstyle=\color[rgb]{0.205, 0.142, 0.73},
%        \lstdefinestyle{C++}{language=C++,style=numbers}’.
}

\begin{document}
  \begin{LARGE}
   Nombre: Christofer Fabián Chávez Carazas
  \end{LARGE}
  \par
  \textbf{Ejercicio 2.2.10} In MATLAB you can type $A = hilb (7)$ to get the $7 x 7$ Hilbert matrix, for
example. Type $help cond$ to find out how to use MATLAB's condition number
function. Use it to calculate $k_{1} (H_{n}), k_{2} (H_{n}) $ and $k_{\infty} (H_{n})$ for $n = 3,6, 9, $ and $ 12$.

\vspace{5mm}

\begin{large}
  \textbf{Comando}
\end{large}

\par

El comando de MATLAB que se usa para obtener los resultados tiene la siguiente estructura
$$ \textbf{cond(hlib(n),P)} $$
Donde, en nuestro caso, $P = \{1,2,\infty\}$.

\begin{enumerate}

\item \textbf{n = 3}
\begin{itemize}
\item $k_{1}(H_{3}) = 748$ 
\item $k_{2}(H_{3}) = 524$
\item $k_{\infty}(H_{3}) = 748$
\end{itemize}

\item \textbf{n = 6}
\begin{itemize}
\item $k_{1}(H_{6}) = 2.9070e+07$ 
\item $k_{2}(H_{6}) = 1.4951e+07$
\item $k_{\infty}(H_{6}) = 2.9070e+07$
\end{itemize}

\item \textbf{n = 9}
\begin{itemize}
\item $k_{1}(H_{9}) = 1.0996e+12$ 
\item $k_{2}(H_{9}) = 4.9315e+11$
\item $k_{\infty}(H_{9}) = 1.0996e+12$
\end{itemize}

\item \textbf{n = 12}
\begin{itemize}
\item $k_{1}(H_{12}) = 3.8273e+16$ 
\item $k_{2}(H_{12}) = 1.7515e+16$
\item $k_{\infty}(H_{12}) = 3.8273e+16$
\end{itemize}

\end{enumerate}

\end{document}
