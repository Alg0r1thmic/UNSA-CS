\documentclass[a4paper,12pt]{article}
\usepackage[utf8]{inputenc}
\usepackage[spanish]{babel}
\usepackage{color}
\usepackage{parskip}
\usepackage{graphicx}
\usepackage{multirow}
\usepackage{listings}
\usepackage{vmargin}
\graphicspath{ {imagenes/} }
\definecolor{mygreen}{rgb}{0,0.6,0}
\definecolor{lbcolor}{rgb}{0.9,0.9,0.9}
\usepackage{epstopdf}


\setpapersize{A4}
\setmargins{2.5cm}       % margen izquierdo
{1.5cm}                        % margen superior
{16.5cm}                      % anchura del texto
{23.42cm}                    % altura del texto
{10pt}                           % altura de los encabezados
{1cm}                           % espacio entre el texto y los encabezados
{0pt}                             % altura del pie de página
{2cm}     

\lstset{
backgroundcolor=\color{lbcolor},
    tabsize=4,    
%   rulecolor=,
    language=[GNU]C++,
        basicstyle=\tiny,
        aboveskip={1.5\baselineskip},
        columns=fixed,
        showstringspaces=false,
        extendedchars=false,
        breaklines=true,
        prebreak = \raisebox{0ex}[0ex][0ex]{\ensuremath{\hookleftarrow}},
        frame=single,
        showtabs=false,
        showspaces=false,
        showstringspaces=false,
        identifierstyle=\ttfamily,
        keywordstyle=\color[rgb]{0,0,1},
        commentstyle=\color[rgb]{0.026,0.112,0.095},
        stringstyle=\color{red},
        numberstyle=\color[rgb]{0.205, 0.142, 0.73},
%        \lstdefinestyle{C++}{language=C++,style=numbers}’.
}

\begin{document}

\begin{Large}
 NOMBRE: CHRISTOFER FABIÁN CHÁVEZ CARAZAS
\end{Large}


\section{Ejercicio 1}

\begin{lstlisting}

int sum = 0;

void * hilo(void * arg){
	int id = *((int*)arg);
	int sumt = 0;
	for(int i = 0; i < 10; i++){
		int r = rand() % 100 + 1;
		sumt += r;
		printf("Soy el hilo %d, numero generado -> %d\n",id,r);	
		//sleep(rand() % 3 + 1);
		sleep(1);
	}
	printf("Soy el hilo %d, la suma obtenida es %d\n",id,sumt);
	sum += sumt;
}

int main(){
	int n = 3;
	srand(time(NULL));
	pthread_t hilos[n];
	int id[3];
	for(int i = 0; i < n; i++){
		id[i] = i;
		pthread_create(&hilos[i],NULL,hilo,(void*)&id[i]);
	}
	for(int i = 0; i < n; i++){
		pthread_join(hilos[i],NULL);
	}
	printf("Soy el padre, la suma total es %d\n",sum);
}
 
\end{lstlisting}

\section{Ejercicio 2}

\begin{lstlisting}
#include <stdio.h>
#include <stdlib.h>
#include <time.h>
#include <pthread.h>
#include <unistd.h>

pthread_mutex_t mutex;

int sum = 0;

struct Rango{
	int ini;
	int fin;
};

void * hilo(void * arg){
	struct Rango r = *((struct Rango*)arg);
	int sumt = 0;
	for(int i = r.ini; i <= r.fin; i++){
		sumt += i;
	}
	pthread_mutex_lock(&mutex);
	sum += sumt;
	pthread_mutex_unlock(&mutex);
}

int main(){
	int n = 1000000;
	int h = 4;
	struct Rango rangos[h];
	pthread_t hilos[h];
	int pib = n / h;
	int actual = 0;
	for(int i = 0; i < h; i++){
		rangos[i].ini = actual + 1;
		actual += pib;
		rangos[i].fin = actual;
		if(i == h-1) rangos[i].fin = n;
		pthread_create(&hilos[i],NULL,hilo,(void*)&rangos[i]);
	}
	for(int i = 0; i < h; i++){
		pthread_join(hilos[i],NULL);
	}
	printf("La suma total es %d\n",sum);
}
\end{lstlisting}

\section{Ejercicio 3}

\begin{lstlisting}
#include <stdio.h>
#include <stdlib.h>
#include <time.h>
#include <pthread.h>
#include <unistd.h>
#include <string.h>
#include <math.h>

#define sizeID 3
#define sizeDesc 50
#define sizePre 6

typedef struct {
char id[sizeID];
char desc[sizeDesc];
int temp;
char pre[sizePre];
}Tarea;

typedef struct{
Tarea tarea;
pthread_t hilo;
}Hilo;

void * thLectura(void *arg){
	static char str[100];
	printf("Ingrese el archivo\n");
	gets(str);
	FILE * fp = fopen(str,"r");
	if(fp == NULL){
		return NULL;
	} 
	return (void*) &str;
}

int leerFichero(char fichero[], Tarea listaTareas[]){
	FILE * fp = fopen(fichero,"r");
	char id[sizeID];
	char desc[sizeDesc];
	int temp = 0;
	char pre[sizePre];
	for(int i = 0; i < sizeID; i++) id[i] = '\0';
	for(int i = 0; i < sizeDesc; i++) desc[i] = '\0';
	temp = 0;
	for(int i = 0; i < sizePre; i++) pre[i] = '\0';
	int estado = 0;
	int sid = 0;
	int sdesc = 0;
	int spre = 0;
	int slt = 0;
	double stemp = 0;
	char c[100];

	while(feof(fp) == 0){
		char c = fgetc(fp);
		if(c == -1) break;
		if(estado == 0){
			if(c == '-') estado = 1;
			else{
				id[sid] = c;
				sid++;
			}
		}
		else if(estado == 1){
			if(c == '-') estado = 2;
			else{
				desc[sdesc] = c;
				sdesc++;
			}
		}
		else if(estado == 2){
			if(c == '-') estado = 3;
			else if(c == 10){
				estado = 0;
				sid = 0;
				sdesc = 0;	
				spre = 0;
				stemp = 0;
				Tarea t;
				strncpy(t.id,id,sizeID);
				strncpy(t.desc,desc,sizeDesc);
				strncpy(t.pre,pre,sizePre);
				t.temp = temp;
				printf("HOLA\n");
				listaTareas[slt] = t;
				slt++;
				for(int i = 0; i < sizeID; i++) id[i] = '\0';
				for(int i = 0; i < sizeDesc; i++) desc[i] = '\0';
				temp = 0;
				for(int i = 0; i < sizePre; i++) pre[i] = '\0';
			}
			else{
				temp = temp * pow(10,stemp);
				temp += ((int) (c - 48));
				stemp++;
			}
		}
		else if(estado == 3){
			if(c == 10){
				estado = 0;
				sid = 0;
				sdesc = 0;	
				spre = 0;
				stemp = 0;
				Tarea t;
				strncpy(t.id,id,sizeID);
				strncpy(t.desc,desc,sizeDesc);
				strncpy(t.pre,pre,sizePre);
				t.temp = temp;
				listaTareas[slt] = t;
				slt++;
				for(int i = 0; i < sizeID; i++) id[i] = '\0';
				for(int i = 0; i < sizeDesc; i++) desc[i] = '\0';
				temp = 0;
				for(int i = 0; i < sizePre; i++) pre[i] = '\0';
			}
			else{
				pre[spre] = c;
				spre++;
			}
		}
	}

	fclose(fp);
	return slt;
}

void mostrarTareas(Tarea lista[], int slt){
	for(int i = 0; i < slt; i++){
		printf("%s ",lista[i].id);
		printf("%s ",lista[i].desc);
		printf("%d ",lista[i].temp);
		printf("%s\n", lista[i].pre);
		/*for(int j = 0; j < sizePre; j++){
			char c = lista[i].pre[j];
			printf("%c", c);
		}
		printf("\n");
		*/
	}

}

void *hacerTarea(void * arg){
	Tarea tarea = *((Tarea*) arg);
	printf("La tarea %s ha COMENZADO\n",tarea.desc);
	sleep(tarea.temp);
	printf("La tarea %s ha TERMINADO\n",tarea.desc);
}


int main(){
	void * file = NULL;
	pthread_t comprobador;
	pthread_create(&comprobador,NULL,thLectura,NULL);
	pthread_join(comprobador,&file);
	if(file == NULL){
		printf("EL archivo no existe\n");
		return 0;
	}
	char *str = (char *) file;
	Tarea listaTareas[50];
	char *temp;
	int slt = leerFichero(str,listaTareas);	
	mostrarTareas(listaTareas,slt);
	Hilo hilos[slt];
	for(int i = 0; i < slt; i++){
		Hilo hilotemp;
		hilos[i] = hilotemp;
		hilos[i].tarea = listaTareas[i];
		char temp[sizeID];
		int st = 0;
		for(int k = 0; k < sizeID; k++) temp[k] = '\0';
		for(int j = 0; j < sizePre; j++){
			if(listaTareas[i].pre[j] == '+'){
			#include <stdio.h>
#include <stdlib.h>
#include <time.h>
#include <pthread.h>
#include <unistd.h>
#include <string.h>
#include <math.h>

#define sizeID 3
#define sizeDesc 50
#define sizePre 6

typedef struct {
char id[sizeID];
char desc[sizeDesc];
int temp;
char pre[sizePre];
}Tarea;

typedef struct{
Tarea tarea;
pthread_t hilo;
}Hilo;

void * thLectura(void *arg){
	static char str[100];
	printf("Ingrese el archivo\n");
	gets(str);
	FILE * fp = fopen(str,"r");
	if(fp == NULL){
		return NULL;
	} 
	return (void*) &str;
}

int leerFichero(char fichero[], Tarea listaTareas[]){
	FILE * fp = fopen(fichero,"r");
	char id[sizeID];
	char desc[sizeDesc];
	int temp = 0;
	char pre[sizePre];
	for(int i = 0; i < sizeID; i++) id[i] = '\0';
	for(int i = 0; i < sizeDesc; i++) desc[i] = '\0';
	temp = 0;
	for(int i = 0; i < sizePre; i++) pre[i] = '\0';
	int estado = 0;
	int sid = 0;
	int sdesc = 0;
	int spre = 0;
	int slt = 0;
	double stemp = 0;
	char c[100];

	while(feof(fp) == 0){
		char c = fgetc(fp);
		if(c == -1) break;
		if(estado == 0){
			if(c == '-') estado = 1;
			else{
				id[sid] = c;
				sid++;
			}
		}
		else if(estado == 1){
			if(c == '-') estado = 2;
			else{
				desc[sdesc] = c;
				sdesc++;
			}
		}
		else if(estado == 2){
			if(c == '-') estado = 3;
			else if(c == 10){
				estado = 0;
				sid = 0;
				sdesc = 0;	
				spre = 0;
				stemp = 0;
				Tarea t;
				strncpy(t.id,id,sizeID);
				strncpy(t.desc,desc,sizeDesc);
				strncpy(t.pre,pre,sizePre);
				t.temp = temp;
				printf("HOLA\n");
				listaTareas[slt] = t;
				slt++;
				for(int i = 0; i < sizeID; i++) id[i] = '\0';
				for(int i = 0; i < sizeDesc; i++) desc[i] = '\0';
				temp = 0;
				for(int i = 0; i < sizePre; i++) pre[i] = '\0';
			}
			else{
				temp = temp * pow(10,stemp);
				temp += ((int) (c - 48));
				stemp++;
			}
		}
		else if(estado == 3){
			if(c == 10){
				estado = 0;
				sid = 0;
				sdesc = 0;	
				spre = 0;
				stemp = 0;
				Tarea t;
				strncpy(t.id,id,sizeID);
				strncpy(t.desc,desc,sizeDesc);
				strncpy(t.pre,pre,sizePre);
				t.temp = temp;
				listaTareas[slt] = t;
				slt++;
				for(int i = 0; i < sizeID; i++) id[i] = '\0';
				for(int i = 0; i < sizeDesc; i++) desc[i] = '\0';
				temp = 0;
				for(int i = 0; i < sizePre; i++) pre[i] = '\0';
			}
			else{
				pre[spre] = c;
				spre++;
			}
		}
	}

	fclose(fp);
	return slt;
}

void mostrarTareas(Tarea lista[], int slt){
	for(int i = 0; i < slt; i++){
		printf("%s ",lista[i].id);
		printf("%s ",lista[i].desc);
		printf("%d ",lista[i].temp);
		printf("%s\n", lista[i].pre);
		/*for(int j = 0; j < sizePre; j++){
			char c = lista[i].pre[j];
			printf("%c", c);
		}
		printf("\n");
		*/
	}

}

void *hacerTarea(void * arg){
	Tarea tarea = *((Tarea*) arg);
	printf("La tarea %s ha COMENZADO\n",tarea.desc);
	sleep(tarea.temp);
	printf("La tarea %s ha TERMINADO\n",tarea.desc);
}


int main(){
	void * file = NULL;
	pthread_t comprobador;
	pthread_create(&comprobador,NULL,thLectura,NULL);
	pthread_join(comprobador,&file);
	if(file == NULL){
		printf("EL archivo no existe\n");
		return 0;
	}
	char *str = (char *) file;
	Tarea listaTareas[50];
	char *temp;
	int slt = leerFichero(str,listaTareas);	
	mostrarTareas(listaTareas,slt);
	Hilo hilos[slt];
	for(int i = 0; i < slt; i++){
		Hilo hilotemp;
		hilos[i] = hilotemp;
		hilos[i].tarea = listaTareas[i];
		char temp[sizeID];
		int st = 0;
		for(int k = 0; k < sizeID; k++) temp[k] = '\0';
		for(int j = 0; j < sizePre; j++){
			if(listaTareas[i].pre[j] == '+'){
				for(int k = 0; k < i; k ++){
					if(strcmp(hilos[k].tarea.id,temp) == 0){
						pthread_join(hilos[k].hilo,NULL);
						break;
					}
				}
				st = 0;
				for(int k = 0; k < sizeID; k++) temp[k] = '\0';
			}
			else if(listaTareas[i].pre[j] == '\0'){
				for(int k = 0; k < i; k ++){
					if(strcmp(hilos[k].tarea.id,temp) == 0){
						pthread_join(hilos[k].hilo,NULL);
						break;
					}
				}
				st = 0;
				for(int k = 0; k < sizeID; k++) temp[k] = '\0';
				break;
			}
			else{
				temp[st] = listaTareas[i].pre[j];
				st++;
			}
		}
		pthread_create(&hilos[i].hilo,NULL,hacerTarea,(void*)&listaTareas[i]);
	}
	for(int i = 0; i < slt; i++){
		pthread_join(hilos[i].hilo,NULL);
	}
	//printf("%s\n", temp);
	
}	for(int k = 0; k < i; k ++){
					if(strcmp(hilos[k].tarea.id,temp) == 0){
						pthread_join(hilos[k].hilo,NULL);
						break;
					}
				}
				st = 0;
				for(int k = 0; k < sizeID; k++) temp[k] = '\0';
			}
			else if(listaTareas[i].pre[j] == '\0'){
				for(int k = 0; k < i; k ++){
					if(strcmp(hilos[k].tarea.id,temp) == 0){
						pthread_join(hilos[k].hilo,NULL);
						break;
					}
				}
				st = 0;
				for(int k = 0; k < sizeID; k++) temp[k] = '\0';
				break;
			}
			else{
				temp[st] = listaTareas[i].pre[j];
				st++;
			}
		}
		pthread_create(&hilos[i].hilo,NULL,hacerTarea,(void*)&listaTareas[i]);
	}
	for(int i = 0; i < slt; i++){
		pthread_join(hilos[i].hilo,NULL);
	}
	//printf("%s\n", temp);
	
}
\end{lstlisting}


\end{document}
