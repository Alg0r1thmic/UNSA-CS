\documentclass[a4paper,12pt]{article}
\usepackage[utf8]{inputenc}
\usepackage[spanish]{babel}
\usepackage{color}
\usepackage{parskip}
\usepackage{graphicx}
\usepackage{multirow}
\usepackage{listings}
\usepackage{vmargin}
\usepackage{datetime}
\newdate{date}{28}{12}{2017}
\graphicspath{ {imagenes/} }
\definecolor{mygreen}{rgb}{0,0.6,0}
\definecolor{lbcolor}{rgb}{0.9,0.9,0.9}
\usepackage{epstopdf}
\usepackage{float}


\setpapersize{A4}
\setmargins{2.5cm}       % margen izquierdo
{1.5cm}                        % margen superior
{16.5cm}                      % anchura del texto
{23.42cm}                    % altura del texto
{10pt}                           % altura de los encabezados
{1cm}                           % espacio entre el texto y los encabezados
{0pt}                             % altura del pie de página
{2cm}     

\lstset{
backgroundcolor=\color{lbcolor},
    tabsize=4,    
%   rulecolor=,
    language=[GNU]C++,
        basicstyle=\tiny,
        aboveskip={1.5\baselineskip},
        columns=fixed,
        showstringspaces=false,
        extendedchars=false,
        breaklines=true,
        prebreak = \raisebox{0ex}[0ex][0ex]{\ensuremath{\hookleftarrow}},
        frame=single,
        showtabs=false,
        showspaces=false,
        showstringspaces=false,
        identifierstyle=\ttfamily,
        keywordstyle=\color[rgb]{0,0,1},
        commentstyle=\color[rgb]{0.026,0.112,0.095},
        stringstyle=\color{red},
        numberstyle=\color[rgb]{0.205, 0.142, 0.73},
%        \lstdefinestyle{C++}{language=C++,style=numbers}’.
}


\begin{document}
\title{Examen Final}
\author{
Christofer Fabián Chávez Carazas \\
\small{Universidad Nacional de San Agustín de Arequipa} \\
\small{Escuela Profesional de Ciencia de la Computación} \\
\small{Aspectos Sociales de la Computación}
}
\date{\displaydate{date}}

\maketitle

\begin{enumerate}
 \item \textbf{Opinión sobre la carta abierta en contra de ciertos desarrollo de la inteligencia artificial.}
 
 \textit{Kill Command} es una película británica del 2016, la historia gira en torno a un grupo de soldados que van a un campo de entrenamiento aislado a ser testigos del funcionamiento
 de una nueva arma. Lo que ellos no saben es que la nueva arma aprende las tácticas utilizadas por ellos y las mejora. La nueva arma se llama SAR (\textit{Study Analise Reprogram}),
 una inteligencia artificial que estudia y aprende tácticas de guerra. Una de las razones por las que crearon el SAR es porque iba a salvar las vidas de
 los soldados en batalla, ya que ya no tendrían que ir a la guerra. Pero un error de programación ocasiona que el SAR se salga de control y empiece a matar al grupo de
 soldados para poder ``aprender'' más tácticas de guerra. En este ejemplo la IA no adquiere conciencia y quiere matar a toda la humanidad, sino que simplemente el programador
 se equivoca en la programación y la IA empieza a matar descontroladamente para aprender. \\
 Actualmente el estado del arte de la inteligencia artificial todavía esta en una etapa temprana con factores que aún no sabemos cómo controlar, como por ejemplo la seguridad. En la película
 \textit{Automata} una IA superdesarrollada programa el sistema de seguridad de las IAs para protegerlas de la gente. Una IA armada no esta libre de ser hackeada
 y utilizada para fines diferentes a los que fue creada.
 En resumen, factores como el error humano, el no saber el alcance de la inteligencia de una IA, la falta de seguridad computacional para IAs, hacen ver que 
 aún no estamos preparados para crear inteligencia artificial avanzada.
 
 \item \textbf{¿A favor o en contra de la neutralidad en la Red?}
 
 Aunque aún no suframos los efectos de la eliminación de la neutralidad de la red, para nosotros los usuarios puede ser algo perjudicial. Las grandes empresas de
 comunicaciones pueden manejar el flujo de internet a su antojo. Por ejemplo, llevado a nivel del Perú, la empresa Claro puede restringir la página de la empresa Movistar y
 viceversa.
 Para nosotros los Científicos de Computación, eliminar la neutralidad de la red puede ser algo que no nos beneficie, más si queremos emprender nuestra propia empresa.
 En el ámbito del software, las empresas casi siempre empiezan como start ups con pocos recursos. Si su idea de software es original e interesante, pero necesita de un servidor
 e internet para ofrecer sus servicios, entonces va a tener muchos problemas para mantenerse y prosperar, como lo tuvo Facebook en sus inicios. Otro posible contra que puede
 surgir al eliminar la neutralidad de la red es un posible monopolio. Por ejemplo, si una empresa de software en surgimiento con una buena aplicación que pueda hacerle la
 competencia a otra empresa más grande, la empresa grande puede pagarle a las grandes empresas de telecomunicaciones para que, prácticamente, bloqueen o limiten los datos
 enviados y recibidos por la aplicación en surgimiento. \\
 Hoy por hoy existe una economía que se está formando en base a la internet: las criptomonedas. Explicado de forma general, las criptomonedas son formas de cambio por 
 procesamiento computacional, y su costo es muy volátil a tal punto que puedes duplicar el dinero invertido en cuestión de días. Eliminar la neutralidad de la red
 crearía un cambio importante en esta economía haciendo que, tal vez, suban su valor hasta límites inimaginables o que simplemente se desplomen hasta el suelo hasta su extinción. \\
 En resumen, estoy a favor de la neutralidad de la red eliminaría muchos beneficios que actualmente tiene la internet, y quitaría oportunidades de emprendimiento 
 en el ámbito software.
 
 \item \textbf{¿Protección y privacidad de datos o vigilancia? El caso de Tor.}
 
 Si hablamos de el caso de Tor estamos hablando también de la Deep Web. Esta parte escondida de la web es un claro ejemplo de lo que las personas hacen cuando
 tienen privacidad y anonimato total. Aquí se pueden encontrar páginas en donde se vende todo tipo de cosas, armas, drogas, asesinos, personas y pornografía ilegal. La
 moneda de cambio por excelencia en estos lugares es el BitCoin por la propiedad de ser irrastreable. Si dejamos que las personas tengan anonimato total, entonces toda
 la red se descontrola a un punto de no retorno. \\
 Si hablamos de la vigilancia, hay un buen ejemplo de esto en la cultura popular. La serie \textit{Person of Interest} habla mucho de la vigilancia de las personas.
 Aquí el gobierno de estados unidos crea una IA que tiene acceso a todos los datos de las personas del mundo y a todas las cámaras del mundo. Su única tarea es
 vigilar a todas las personas del mundo y encontrar a personas que puedan atentar contra el gobierno (terroristas). Lo interesante de esto es que las personas no
 tienen ni idea de que son vigiladas. Llegar a este punto de vigilancia ya es algo excesivo. Se debe buscar un equilibrio entre el anonimato total y la vigilancia total.
 
\end{enumerate}



\end{document}

