\documentclass{beamer}
\usepackage[utf8]{inputenc}
\usepackage[spanish,es-tabla]{babel}
\usepackage{graphicx}
\usepackage{subfigure} % subfiguras
\usepackage{caption}
\usepackage{pgfplots}
\graphicspath{ {imagenes/} }

\usetheme{PaloAlto}

\title{Calidad de Sistemas Web}

\author{Christofer Chávez Carazas \and Ruben Torres Lima}


\institute[Universidad Nacional de San Agustín] 


\date{\today}

\subject{Calidad de Software}
\AtBeginSubsection[]
{
  \begin{frame}<beamer>{Outline}
    \tableofcontents[currentsection,currentsubsection]
  \end{frame}
}

\begin{document}

\begin{frame}
  \titlepage
\end{frame}

\begin{frame}{Calidad y Usabilidad}
  \begin{itemize}
   \item La ISO 9000 asocia la calidad del producto con que este satisfará los requisitos establecidos (calidad de manufacturación).
   \item La ISO 9126 nos da características: funcionalidad, eficiencia, usabilidad, fiabilidad, mantenibilidad, portabilidad.
   \item Calidad percibida.
  \end{itemize}
\end{frame}

\begin{frame}{Calidad y Usabilidad}
  \begin{itemize}
   \item La ISO 14598-1 nos dice que cuando un producto satisface necesidades implícitas y explícitas en condiciones específicas
   es cuando tiene calidad externa.
   \item La ISO 14598-1 define la calidad de uso como: ``la efectividad, eficiencia y satisfacción con la cual ciertos usuarios específicos pueden alcanzar ciertas metas en entornos
   concretos''.
   \item La ISO 9241-11 define la usabilidad como: ``El instante en el cual un usuario puede ser utilizado por usuarios específicos para alcanzar metas específicas con
   efectividad, eficiencia y satisfacción en un contexto de uso específico''
  \end{itemize}
\end{frame}

\begin{frame}{Usabilidad}
  \begin{itemize}
   \item Efectividad, eficiencia y satisfacción en un contexto específico (ISO 9241-11).
   \item Aprendizaje, eficiencia, necesidad de recordar, errores y satisfacción (J. Nielsen).
   \item Efectividad o relevancia, eficiencia, actitud del usuario, comprensibilidad y seguridad (Fabio Paterno).
  \end{itemize}
\end{frame}

\begin{frame}{Usabilidad}
  \begin{itemize}
   \item En un contexto dado.
   \item Entornos Web.
   \item Accesibilidad.
   \item Usabilidad universal: ``Diseñar productos que sean usables por el rango más amplio de personas, funcionando en el rango más amplio
   de situaciones''
  \end{itemize}
\end{frame}

\begin{frame}{Accesibilidad}
 \begin{itemize}
  \item Equitativo.
  \item Flexible.
  \item Simple e intuitivo.
  \item Información perceptible.
  \item Tolerancia para el error.
  \item Esfuerzo físico mínimo.
  \item Tamaño y espacio para poder aproximarse y usar el diseño.
 \end{itemize}
\end{frame}

\begin{frame}{Usabilidad en el proceso de desarrollo de software}
  \begin{itemize}
   \item Presente en todo el proceso.
   \item Ingeniería de la usabilidad.
   \item Conocer a los usuarios.
   \item Que los usuarios puedan realizar evaluaciones.
   \item Actividad multidisciplinar.
  \end{itemize}
\end{frame}

\begin{frame}{Usabilidad en el proceso de desarrollo de software}
 Cuando se mide la usabilidad de un sistema es necesario tener en cuenta la siguiente información:
 \begin{itemize}
  \item Descripción de los objetivos.
  \item Descripción del contexto.
  \item Los valores reales o perseguidos de efectividad, eficiencia y satisfacción para el contexto deseado.
 \end{itemize}
\end{frame}

\begin{frame}{Contexto}
 \begin{itemize}
  \item Descripción de los usuarios.
  \item Descripción de las tareas.
  \item Descripción del equipo.
  \item Descripción del entorno.
  \item Métricas de usabilidad.
 \end{itemize}
\end{frame}

\begin{frame}{Usabilidad}
 Usabilidad significa que las personas puedan utilizar el sistema de forma rápida y fácil para llevar a cabo sus tareas.
  \begin{itemize}
   \item Usabilidad significa centrarse en el usuario.
   \item Las personas utilizan productos para ser más productivos.
   \item Los usuarios son gente ocupada que intenta llevar a cabo sus tareas.
   \item Son los usuarios los que deciden cuándo un sistema es fácil de utilizar.
  \end{itemize}
\end{frame}

 


\begin{frame}{Desarrollo orientado al usuario}
  \begin{itemize}
   \item La participación activa del usuario.
   \item Distribución apropiada de funciones entre los usuarios y la tecnología.
   \item Retroalimentación del usuario.
   \item Diseño multidisciplinar.
  \end{itemize}
\end{frame}

\begin{frame}{IDEAS}
  \begin{itemize}
   \item Requisitos.
   \item Análisis.
   \item Diseño.
   \item Implementación.
  \end{itemize}
\end{frame}

\begin{frame}{IDEAS - Requisitos}
 \begin{itemize}
  \item Modelo de tareas: definición detallada de las tareas.
  \begin{block}{Modelo de usuario: conocer al usuario}
   \begin{itemize}
    \item Diferentes tipos de usuario.
    \item La proyección sobre las acciones que dentro de una tarea puede realizar el usuario.
    \item La forma más adecuada de mostrar la información.
   \end{itemize}
  \end{block}
  \item Nivel de sistema.
  \item Nivel de tarea.
  \item Nivel de usuario.
  \item Nivel de usuario-tarea.
 \end{itemize}
\end{frame}

\begin{frame}{IDEAS - Diseño}
  \begin{block}{Modelo de diálogo}
   \begin{itemize}
    \item Se toma en cuenta el modelo de usuario.
    \item Se le ofrece al usuario las herramientas más adecuadas para que pueda ejecutar las acciones permitidas dentro de las tareas a las
    que tiene acceso.
    \item De forma abstracta.
   \end{itemize}
  \end{block}
\end{frame}

\begin{frame}{IDEAS - Implementación}
 \begin{block}{Modelo de presentación}
  \begin{itemize}
   \item Cómo presentar el modelo de diálogo con objetos concretos.
   \item Se siguen guías de estilo y heurísticas.
  \end{itemize}
 \end{block}
\end{frame}

\begin{frame}{Medida de la usabilidad}
 \begin{itemize}
  \item Criterios últimos, son complicados de medir.
  \item Criterios actuales, si se pueden medir.
  \item Métodos formativos frente a sumativos.
  \item Métodos de descubrimiento frente a métodos de decisión.
  \item Métodos formales e informales.
  \item Métodos que involucran al usuario y métodos que no lo hacen.
  \item Métodos completos frente a métodos componente.
 \end{itemize}
\end{frame}

\begin{frame}{Ejemplos}
 \begin{itemize}
  \item El SUMI(Software Usability Measurement Inventory)
  \item El MUMMS(Measuring the Usability of Multi-Media Systems)
  \item El WAMMI(Website Analysis and Measurement Inventory)
  \item El SUS(System Usability Scale)
 \end{itemize}
\end{frame}


\begin{frame}{Conclusiones}
 \begin{itemize}
  \item En la perspectiva del usuario, hablar de calidad es hablar de usabilidad.
  \item IDEAS pretende integrar el diseño de interfaz en el proceso de desarrollo de software en base a las ideas propuestas
  en torno a la ingeniería de la usabilidad.
  \item La evaluación siempre es constante, no se debe dejar para el final.
  \item El reto es alcanzar verdaderas interfaces invisibles.
 \end{itemize}
\end{frame}









\end{document}


