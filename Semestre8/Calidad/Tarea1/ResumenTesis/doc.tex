\documentclass[a4paper,12pt]{article}
\usepackage[utf8]{inputenc}
\usepackage[spanish]{babel}
\usepackage{color}
\usepackage{parskip}
\usepackage{graphicx}
\usepackage{multirow}
\usepackage{listings}
\usepackage{vmargin}
\usepackage{datetime}
\newdate{date}{5}{09}{2017}
\graphicspath{ {imagenes/} }
\definecolor{mygreen}{rgb}{0,0.6,0}
\definecolor{lbcolor}{rgb}{0.9,0.9,0.9}
\usepackage{epstopdf}


\setpapersize{A4}
\setmargins{2.5cm}       % margen izquierdo
{1.5cm}                        % margen superior
{16.5cm}                      % anchura del texto
{23.42cm}                    % altura del texto
{10pt}                           % altura de los encabezados
{1cm}                           % espacio entre el texto y los encabezados
{0pt}                             % altura del pie de página
{2cm}     



\begin{document}
\title{Resumen Capítulo 2 y 3}
\author{
Christofer Fabián Chávez Carazas \\
\small{Universidad Nacional de San Agustín de Arequipa} \\
\small{Escuela Profesional de Ciencia de la Computación} \\
\small{Calidad de Software}
}
\date{\displaydate{date}}

\maketitle

\begin{enumerate}
 \item \textbf{Modelo de Calidad de Procesos de Software}
 
 Cuando se habla de calidad, pueden surgir varios conceptos dependiendo del contexto en el que se trabaje, por ejemplo, W. Edwards Deming
 nos dice que ``La calidad es una serie de cuestionamientos hacia una mejora continua, definidos en términos de satisfacción del cliente'', y
 Joseph M. Duran nos dice que la calidad es ``La adecuación para el uso, satisfaciendo las necesidades del cliente''. Y así se podrían nombrar a varios
 autores. En el ámbito del software, es necesario diferenciar entre la calidad del producto de software y la calidad del proceso de desarrollo.
 El primero depende mucho del segundo, ya que ``Sin un buen proceso de desarrollo es casi imposible obtener un buen producto''. \\
 El mayor problema que se presenta al hablar de calidad de software es encontrar propiedades en el software que nos den una indicación de su calidad.
 Para resolver esto surgen los modelos de calidad, que son puntos de referencia reproducibles. También existen los modelos de calidad de procesos, los cuales
 buscan alcanzar una mejora de proceso que, según Sommerville, significa entender los procesos existentes y cambiarlos para reducir los costes y el tiempo
 de desarrollo.
 
 \item \textbf{Modelos de Evaluación y Mejora de Procesos de Software}
  
 \begin{itemize}
  \item \textbf{CMM:} El modelo \textit{Capability Maturity Model}, contiene los elementos para conseguir proceses eficaces en uno o
  más cuerpos de conocimiento. Se establecen 5 niveles de capacidad ala hora de medir la madurez de los procesos de una organización, siendo el nivel inicial caótico
  y el final maduro. Para que una organización pueda estar en un nivel tiene que satisfacer criterios relacionados a ese nivel y al nivel anterior.
  \item \textbf{SCE:} Tiene como objetivo determinar la capacidad de una organización en base a sus procesos de software. Usa CMM como modelo de referencia.
  \item \textbf{CBA-IPI (CMM-Based): } Le permite a la organización conocer la capacidad de sus procesos de software mediante la identificación de fortalezas y debilidades en base al modelos CMM.
  \item \textbf{IDEAL: } Propone el camino de acciones que debe tener el programa de mejora de procesos de software cuando una organización desea implementar CMM, en el cual se basa.
  \item \textbf{Personal Software Process (PSP): } Orientado a la mejora individual de cada ingeniero de software, asistiéndolos en la realización de planes precisos.
  \item \textbf{Team Software Process (TSP): } Ayuda a conformar equipos para el desarrollo de software de calidad en base a PSP. Se busca minimizar el tiempo del proyecto y maximizar
  la productividad.
  \item \textbf{People-CMM: } Ayuda a las organizaciones a resolver de forma exitosa los aspectos críticos relacionados con sus recursos humanos. El objetivo es mejorar los procesos
  de gestión y desarrollo de los empleados.
  \item \textbf{CMMI: } El modelo \textit{Capability Maturity Model Integration} Proporciona un enfoque disciplinado para mejorar los procesos de una organización. A diferencia de CMM, CMMI se aplica a lo largo de
  todo el ciclo de vida de ingeniería, no únicamente al ciclo de vida del desarrollo del producto software. CMMI usa dos tipos de representaciones de sus modelos, representación por etapas y continuo.
  La representación por etapas proporciona un marco predefinido para la mejora organizacional basada en el agrupamiento y ordenación de procesos y en las relaciones organizacionales asociadas. A las etapas se les denomina nivel de madurez.
  CMMI contempla cinco niveles de madurez: inicial, gestionado, definido, cuantitativamente gestionado y optimizado.
  La representación continua proporciona una guía en la cual debería realizarse el proceso de mejora.
  \item \textbf{SCAMPI: } es un método de evaluación de clase A, aplicable a un amplio rango de modelos de evaluación. Basado en CMM-Based, EIA/IS y SCE.
  Las principales fases de evaluación son:
  \begin{itemize}
   \item Planificación y preparación de la evaluación.
   \item realización de la evaluación.
   \item Informe de resultados.
  \end{itemize}
  \item \textbf{SPICE :} Representa el mayor marco de colaboración internacional establecida con la finalidad de desarrollar un estándar de evaluación de procesos de software.
  \item \textbf{ISO/IEC 15504: } Es un estándar aplicable a cualquier organización que quiera mejorar la capacidad de sus procesos, independientemente de la organización y todo lo que ella involucra.
  \item \textbf{Bootstrap: } Resultado del proyecto ESPRIT 5451. Describe el proceso de evaluación desarrollados para determinar si una organización se encuentra en cierto nivel de madurez, 
  identificando los puntos débiles y fuertes.
  \item \textbf{Familia ISO 9000: } tiene como objetivo establecer los principios a seguir sobre sistemas de calidad y el aseguramiento de las mismas en una organización.
  Las normas ISO 9001, 9002 y 9003 se unificaron en una sola, tomando el nombre de ISO 9001:2000.
  \begin{itemize}
   \item \textbf{ISO 9001:2000: } Especifica los requisitos para un sistema de gestión de calidad cuando una organización necesita productos que satisfagan los requisitos del cliente y aspira 
   a aumentar la satisfacción del cliente.
   \item \textbf{ISO 9001:2008: } Su objetivo es proporcionar un conjunto de requisitos, que si se implementan de manera adecuada, darán confianza a sus clientes.
   \item \textbf{La guía TickIT: } Creada por el DTI del Reino Unido por el desinterés en adoptar la norma ISO 9000 de las empresas de desarrollo.
   \item \textbf{ISO/IEC 9003:2004: } Ofrece una guía de la aplicación de la ISO 9001:2000 en empresas desarrolladoras de software.
   \item \textbf{ISO/IEC 12207:2004: } Establece un marco común orientado a los procesos del ciclo de vida del software.
  \end{itemize}
  \item \textbf{COMPETISOFT: } Es una iniciativa integradora de diferentes propuestas de mejora de procesos software para PYMES desarroladoras de software.
  El objetivo es incrementar el nivel de competitividad de las PYMES desarrolladoras de software en iberoamérica.
  \item \textbf{MPS:BR} Mejora de proceso del Software Brasileño. Liderado por la empresa SOFTEX de Brasil. El modelo contempla 22 procesos de la norma ISO 12207 y 7 niveles de madurez.
  El método de evaluación se basa en la norma ISO 15504.
  \item \textbf{MoProSoft: } Surge como respuesta a la dificultad de las empresas por implementar los diferentes modelos de mejora de procesos, siendo un modelo fácil de entender, práctico y económico.
  El objetivo de MoProSoft es elevar la capacidad de las organizaciones para ofrecer servicios con calidad.
  \item \textbf{EvalProSoft: } Es un método de evaluación de procesos para la industria de software. Este método de evaluación se aplica a las organizaciones 
  dedicadas al desarrollo y/o mantenimiento de software.
  \item \textbf{Agile SPI: } Es un framework de SPI basado en modelos livianos que tiene como objetivo guiar la mejora de los procesos de desarrollo de software de forma ágil.
  \item \textbf{SIMEP-SW: } Proyecto financiado por Colciencias y la Universidad del Cauca en Colombia. Intenta cubrir dos esfuerzos: el de aliviar requisitos y guiar en el procesos de mejora, así como
  el de generar un conjunto de recomendaciones prácticas para la implementación de los requisitos del proceso software.
  \item \textbf{PMCOMPETISOFT: } es un proceso para guiar la mejora de procesos software en pequeñas empresas. Proporciona una guía, paso a paso, para llevar a cabo esfuerzos de 
  mejora de procesos.
  \item \textbf{VSE: } El proyecto VSE es un esfuerzo internacional focalizado en organizaciones pequeñas. La ISO/IEC 29110 tiene como propósito contribuir al incremento de la competitividad de la VSE.
  \item \textbf{COMPETISOFT - PUCP: } Es un esfuerzo continuo que la Pontificia Universidad Católica del Perú lleva a cabo a favor de la industria del software. En el proyecto se apuesta por
  el mejoramiento sostenido de los procesos de software en base a un modelo que se ajuste a las necesidades de las organizaciones vinculadas al desarrollo de software.
  
 \end{itemize}

 
\end{enumerate}


\end{document}

