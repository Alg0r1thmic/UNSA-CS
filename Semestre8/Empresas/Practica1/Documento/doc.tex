\documentclass[a4paper,12pt]{article}
\usepackage[utf8]{inputenc}
\usepackage[spanish]{babel}
\usepackage{color}
\usepackage{parskip}
\usepackage{graphicx}
\usepackage{multirow}
\usepackage{listings}
\usepackage{vmargin}
\usepackage{datetime}
\newdate{date}{23}{08}{2017}
\graphicspath{ {imagenes/} }
\definecolor{mygreen}{rgb}{0,0.6,0}
\definecolor{lbcolor}{rgb}{0.9,0.9,0.9}
\usepackage{epstopdf}


\setpapersize{A4}
\setmargins{2.5cm}       % margen izquierdo
{1.5cm}                        % margen superior
{16.5cm}                      % anchura del texto
{23.42cm}                    % altura del texto
{10pt}                           % altura de los encabezados
{1cm}                           % espacio entre el texto y los encabezados
{0pt}                             % altura del pie de página
{2cm}     



\begin{document}
\title{Práctica de Empresas}
  \author{
  Christofer Fabián Chávez Carazas \\
  \small{Universidad Nacional de San Agustín} \\
  \small{Empresas I}
}
\date{\displaydate{date}}

\maketitle


\begin{itemize}
 \item \textbf{¿Cuál es la región más innovadora del mundo?}
 La región más innovadora del mundo tendría que estar bien en el ámbito científico, ya que es lo necesario para que una
 región avance y sea considerada innovadora. En mi opinión, uno de los países más innovadores es Estados Unidos. Y si hablamos de continentes, vendría ser Europa.
 
 \item \textbf{¿Áreas prioritarias en Ciencia y Tecnología en el Perú?} \\
 En estos momentos las áreas prioritarias serían las relacionadas con resolver problemas generales de la población,
 como por ejemplo, el exceso de tráfico en las grandes ciudades o los desastres naturales. También es importante el área
 administrativa del estado. Hace falta mucho software que administre las diferentes áreas del estado, para tener una mejor
 visión de lo que está pasando en el Perú en tiempo real, para luego dar lugar a que nuestros gobernantes tomen mejores decisiones.
 CONCYTEC nos presenta el plan nacional  de CTI 2006 – 2021 que tiene como objetivo asegurar la articulación y concertación entre los
 actores del SINACYT, enfocando sus esfuerzos para atender las demandas tecnológicas en áreas estratégicas prioritarias, con la finalidad de 
 elevar el valor agregado y la competitividad, mejorar la calidad de vida de la población y contribuir con el manejo responsable del medio ambiente.
 
 \item \textbf{¿Cuáles son los motivos para plantearse la creación de una empresa?} \\
 En mi opinión puede haber varios motivos por lo cual uno quiera crear una empresa. Uno de ellos, y por el que tal vez todas las
 empresas piensan, es para lucrar. Otro motivo, y tal vez el más importante, es producir un producto o servicio que resuelva 
 un problema del consumidor, o que haga que su día a día sea más fácil. Además se pueden mencionar otros motivos como que hay otras
 personas interesadas en emprender contigo, o tienes experiencia en el campo en el que van a emprender, o también por razones personales.
 
 \item \textbf{¿Qué elementos principales conforman una empresa?} \\
 Se consideran elementos de la empresa todos aquellos factores, tanto internos como externos, que influyen directa o indirectamente en su funcionamiento.
 Los principales elementos de la empresa son los siguientes:
 \begin{itemize}
    \item \textbf{El empresario:} Es la persona o conjunto de personas encargadas de gestionar y dirigir tomando las decisiones necesarias para la buena marcha de la empresa. No siempre coinciden la figura del empresario y la del propietario, puesto que se debe diferenciar el director, que administra la empresa, de los accionistas y propietarios que han arriesgado su dinero percibiendo por ello los beneficios.
    \item \textbf{Los trabajadores:} Es el conjunto de personas que rinden su trabajo en la empresa, por lo cual perciben unos salarios.
    \item \textbf{La tecnología:} Está constituida por el conjunto de procesos productivos y técnicas necesarias para poder fabricar (técnicas, procesos, máquinas, ordenadores, etc.).
    \item \textbf{Los proveedores:} Son personas o empresas que proporcionan las materias primas, servicios, maquinaria, etc., necesarias para que las empresas puedan llevar a cabo su actividad.
 \end{itemize}
 
 \item \textbf{¿Qué características posee una empresa de Base Tecnológica?} \\
 Se denominan Empresas de Base Tecnológica (EBTs) aquellas que basan su actividad en las aplicaciones de nuevos descubrimientos científicos o tecnológicos para la generación de nuevos productos, procesos o servicios.
 La importancia de estas empresas para potenciar el tejido tecnológico y el desarrollo económico, favorecer la creación de empleo de alta cualificación, aportando un alto valor añadido al entorno industrial ha hecho que las universidades y otras instituciones públicas de investigación les dediquen una creciente atención como auténticos motores en la transferencia de conocimiento.
 En el modelo europeo muchas de estas empresas han surgido desde las universidades y organismos públicos de I+D y se denominan generalmente "spin-off". Son empresas caracterizadas por tener una fuerte base tecnológica y generalmente alta carga de innovación. Representan una vía muy importante para la transferencia de los resultados de investigación, un beneficio para la sociedad debido a la posibilidad de acceder a nuevos productos o servicios y favorecen la inserción de los jóvenes en el mundo laboral.
 
 \item \textbf{¿Tiene alguna idea de negocio a emprender?¿Cuál?} \\
 Aún no. Sigo pensando.
 
 \item \textbf{¿Tiene alguna experiencia como empresario o gerente?} \\
 No tengo experiencia como empresario o gerente.
 
\end{itemize}

\end{document}

