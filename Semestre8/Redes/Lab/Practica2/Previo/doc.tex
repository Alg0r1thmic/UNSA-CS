\documentclass[a4paper,12pt]{article}
\usepackage[utf8]{inputenc}
\usepackage[spanish]{babel}
\usepackage{color}
\usepackage{parskip}
\usepackage{graphicx}
\usepackage{multirow}
\usepackage{listings}
\usepackage{vmargin}
\usepackage{datetime}
\newdate{date}{07}{09}{2017}
\graphicspath{ {imagenes/} }
\definecolor{mygreen}{rgb}{0,0.6,0}
\definecolor{lbcolor}{rgb}{0.9,0.9,0.9}
\usepackage{epstopdf}


\setpapersize{A4}
\setmargins{2.5cm}       % margen izquierdo
{1.5cm}                        % margen superior
{16.5cm}                      % anchura del texto
{23.42cm}                    % altura del texto
{10pt}                           % altura de los encabezados
{1cm}                           % espacio entre el texto y los encabezados
{0pt}                             % altura del pie de página
{2cm}     


\begin{document}
\title{Previo de la Segunda Práctica}
\author{
Christofer Fabián Chávez Carazas \\
\small{Universidad Nacional de San Agustín de Arequipa} \\
\small{Escuela Profesional de Ciencia de la Computación} \\
\small{Computación Centrada en Redes}
}
\date{\displaydate{date}}

\maketitle

Definir claramente los siguientes términos:

\begin{itemize}
 \item \textbf{Controlador: } Programa que permite al sistema operativo interactuar con un periférico, haciendo una 
 abstracción del hardware y proporcionando una interfaz para utilizar el dispositivo.
 \item \textbf{Dirección MAC: } Dirección del hardware de control de acceso a soportes de un distribuidor que identifica los equipos, los
 servidores, los routers u otros dispositivos de red. También conocida como dirección física, tiene 48 bits y es única para cada dispositivo.
 \item \textbf{Dirección IP: } Es un número que identifica exclusivamente un equipo en una red o en Internet. Las direcciones IP de los equipos en una red
 de servidor-cliente pueden cambiar diariamente o con mayor frecuencia. La dirección de un equipo debe ser única dentro de una red.
 \item \textbf{Máscara de Red: } La máscara de red permite distinguir dentro de la dirección IP, los bits que identifican a la red y los bits que identifican al host.
 Dependiendo de la configuración que se tome, se pueden clasificar las direcciones en diferentes clases.
 \item \textbf{Puerta de Enlace: } Es el dispositivo que actúa de interfaz de conexión entre aparatos o dispositivos y también posibilita compartir recursos entre dos o más computadoras.
 \item \textbf{Subred: } Porciones de una red TCP/IP usada para aumentar el ancho de banda en la red subdividiendo la red en porciones o segmentos.
 \item \textbf{Proxy: } Aplicación (o agente) que se ejecuta en una gateway de seguridad y que actúa como servidor y cliente, que acepta conexiones de un cliente y realiza solicitudes en
 nombre de este al servidor de destino. Existen varios tipos de proxy, cada uno de ellos con un objetivo específico.
 \item \textbf{Firewall: } Hardware, software o una combinación de ambos que se usa para evitar que los usuarios de Internet no autorizados accedan a una red privada. Toda la información que entra en o que sale de una
 red debe pasar por un firewall, que examina los paquetes de información y bloquea los que no cumplan criterios de seguridad.
 \item \textbf{Host: } En un entorno de red, una computadora u otro dispositivo conectado a una red que provee y utiliza servicios de ella. 
 \item \textbf{ARP: } (\textit{Address Resolution Protocol}) Es un protocolo de comunicaciones de la capa de enlace, responsable de encontrar la dirección MAC que corresponde a una determinada dirección IP en una red local.
 \item \textbf{Hostname: } Es un nombre único e informal que se le da a un dispositivo conectado a una red.
 \item \textbf{DHCP: } El programa usado en las redes servidor-cliente, en comparación con las redes punto a punto. Cuando una red usa el DHCP, la red incluye un servidor DHCP que asigna automáticamente las direcciones IP a los equipos en la
 red según lo necesario. Cada vez que un equipo se desconecta de la red y se vuelve a conectar, el servidor DHCP asigna una nueva dirección IP.
\end{itemize}

\begin{thebibliography}{1}

\bibitem{sym}
 Symantec. Glosario.

 
\end{thebibliography}
\end{document}

