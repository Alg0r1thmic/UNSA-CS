\documentclass[a4paper,12pt]{article}
\usepackage[utf8]{inputenc}
\usepackage[spanish]{babel}
\usepackage{color}
\usepackage{parskip}
\usepackage{graphicx}
\usepackage{multirow}
\usepackage{listings}
\usepackage{vmargin}
\usepackage{datetime}
\newdate{date}{22}{12}{2017}
\graphicspath{ {imagenes/} }
\definecolor{mygreen}{rgb}{0,0.6,0}
\definecolor{lbcolor}{rgb}{0.9,0.9,0.9}
\usepackage{epstopdf}
\usepackage{float}


\setpapersize{A4}
\setmargins{2.5cm}       % margen izquierdo
{1.5cm}                        % margen superior
{16.5cm}                      % anchura del texto
{23.42cm}                    % altura del texto
{10pt}                           % altura de los encabezados
{1cm}                           % espacio entre el texto y los encabezados
{0pt}                             % altura del pie de página
{2cm}     

\lstset{
backgroundcolor=\color{lbcolor},
    tabsize=4,    
%   rulecolor=,
    language=[GNU]C++,
        basicstyle=\tiny,
        aboveskip={1.5\baselineskip},
        columns=fixed,
        showstringspaces=false,
        extendedchars=false,
        breaklines=true,
        prebreak = \raisebox{0ex}[0ex][0ex]{\ensuremath{\hookleftarrow}},
        frame=single,
        showtabs=false,
        showspaces=false,
        showstringspaces=false,
        identifierstyle=\ttfamily,
        keywordstyle=\color[rgb]{0,0,1},
        commentstyle=\color[rgb]{0.026,0.112,0.095},
        stringstyle=\color{red},
        numberstyle=\color[rgb]{0.205, 0.142, 0.73},
%        \lstdefinestyle{C++}{language=C++,style=numbers}’.
}


\begin{document}
\title{Tarea: Identificar Puertos}
\author{
Christofer Fabián Chávez Carazas \\
\small{Universidad Nacional de San Agustín de Arequipa} \\
\small{Escuela Profesional de Ciencia de la Computación} \\
\small{Computación Centrada en Redes}
}
\date{\displaydate{date}}

\maketitle

\begin{large}
 \textbf{Enunciado}
\end{large}

  Identificar los protocolos asociados a los siguientes puertos: 22, 66, 107, 118, 119, 137, 138, 150, 194, 443, 522,
6891, 6901 indicando si son TCP o UDP y brevemente la función que soportan.

\begin{large}
 \textbf{Resolución}
\end{large}

\begin{itemize}
 \item \textbf{Puerto 22}
 \begin{itemize}
  \item \textbf{Protocolo Asociado:}  SSH (Secure SHell)
  \item \textbf{Soporta:} TCP y UDP
  \item \textbf{Función:} SSH es un protocolo de administración remota que permite a los usuarios controlar y modificar sus servidores remotos a través de Internet.
  A diferencia de otros protocolos de comunicación remota tales como FTP o Telnet, SSH encripta la sesión de conexión, haciendo imposible que alguien pueda obtener contraseñas no encriptadas.
 \end{itemize}
 \item \textbf{Puerto 66}
 \begin{itemize}
  \item \textbf{Protocolo Asociado:} Oracle SQLNet
  \item \textbf{Soporta:} TCP y UDP
  \item \textbf{Función:} Oracle SQLNet es un software de redes de Oracle que permite el acceso remoto a datos entre programas y la base de datos de Oracle.
 \end{itemize}
 \item \textbf{Puerto 107}
 \begin{itemize}
  \item \textbf{Protocolo Asociado:} Remote Telnet Service
  \item \textbf{Soporta:} TCP y UDP
  \item \textbf{Función:} RTelnet es una versión más segura del Telnet para sistemas Unix. Provee de funcionalidades similares a las de Telnet con la diferencia
  de que los host se encuentran monitoreados por un firewall.
 \end{itemize}
 \item \textbf{Puerto 118}
 \begin{itemize}
  \item \textbf{Protocolo Asociado:} SQL Services
  \item \textbf{Soporta:} TCP y UDP
  \item \textbf{Función:} Es un sistema de manejo de bases de datos del modelo relacional, desarrollado por la empresa Microsoft. El puerto es utilizado por esta aplicación
  para las conexiones remotas a las bases de datos creadas en ella.
 \end{itemize}
 \item \textbf{Puerto 119}
 \begin{itemize}
  \item \textbf{Protocolo Asociado:} NNTP (Network News Transport Protocol)
  \item \textbf{Soporta:} TCP y UDP
  \item \textbf{Función:}  NNTP  es un protocolo inicialmente creado para la lectura y publicación de artículos de noticias en Usenet. El funcionamiento del NNTP es muy
  sencillo, consta de un servidor en el que están almacenadas las noticias y a él se conectan los clientes a través de la red.
  La conexión entre cliente y servidor se hace de forma interactiva consiguiendo así un número de artículos duplicados muy bajo.
 \end{itemize}
 \item \textbf{Puerto 137}
 \begin{itemize}
  \item \textbf{Protocolo Asociado:} NetBios Name Service (NBNS)
  \item \textbf{Soporta:} TCP (raramente) y UDP
  \item \textbf{Función:} NetBios es una especificación de interfaz para acceso a servicios de red, es decir, una capa de software desarrollado para enlazar un sistema operativo
  de red con hardware específico. NBNS es una parte de esta interfaz. Tiene casi el mismo propósito que una DNS: convertir direcciones comerciales a direcciones IP.
 \end{itemize}
 \item \textbf{Puerto 138}
 \begin{itemize}
  \item \textbf{Protocolo Asociado:} NetBios Datagram Service 
  \item \textbf{Soporta:} UDP
  \item \textbf{Función:} Este puerto es utilizado para el envío de datagramas por parte de NetBios
 \end{itemize}
 \item \textbf{Puerto 150}
 \begin{itemize}
  \item \textbf{Protocolo Asociado:} SQL-Net
  \item \textbf{Soporta:} TCP y UDP
  \item \textbf{Función:} Nos permite conectarnos remotamente a bases de datos SQL
 \end{itemize}
 \item \textbf{Puerto 194}
 \begin{itemize}
  \item \textbf{Protocolo Asociado:} IRC (Internet Relay Chat)
  \item \textbf{Soporta:} TCP y UDP
  \item \textbf{Función:} IRC es un protocolo de comunicación en tiempo real basado en texto, que permite debates entre dos o más personas.
  Se diferencia de la mensajería instantánea en que los usuarios no deben acceder a establecer la comunicación de antemano, de tal forma que todos los usuarios
  que se encuentran en un canal pueden comunicarse entre sí, aunque no hayan tenido ningún contacto anterior.
 \end{itemize}
 \item \textbf{Puerto 443}
 \begin{itemize}
  \item \textbf{Protocolo Asociado:} HTTPS
  \item \textbf{Soporta:} TCP
  \item \textbf{Función:} Es un protocolo de aplicación basado en el HTTP, destinado a la transferencia segura de datos de Hipertexto, es decir, es la versión segura de
  HTTP.
 \end{itemize}
 \item \textbf{Puerto 522}
 \begin{itemize}
  \item \textbf{Protocolo Asociado:} NetMeeting
  \item \textbf{Soporta:} TCP
  \item \textbf{Función:} NetMeeting fue un cliente de videollamada VoIP y multipunto incluido en muchas versiones de Microsoft Windows.
 \end{itemize}
 \item \textbf{Puerto 6891}
 \begin{itemize}
  \item \textbf{Protocolo Asociado:} MSN Messenger (archivos) (6891 - 6900)
  \item \textbf{Soporta:} TCP
  \item \textbf{Función:} MSN Messenger fue un programa de mensajería instantánea creado por Microsoft en 1999 y descontinuado en el 2005 debido a su reemplazo
  por Windows Live Messenger. Estos puertos estaban destinados para el envío de archivos y mensajes desde la aplicación.
 \end{itemize}
 \item \textbf{Puerto 6901}
 \begin{itemize}
  \item \textbf{Protocolo Asociado:} MSN Messenger (voz)
  \item \textbf{Soporta:} TCP
  \item \textbf{Función:} Este protocolo se utiliza para el transporte de voz desde la aplicación.
 \end{itemize}














 
\end{itemize}




\end{document}

